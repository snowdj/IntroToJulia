
% Default to the notebook output style

    


% Inherit from the specified cell style.




    
\documentclass[11pt]{article}

    
    
    \usepackage[T1]{fontenc}
    % Nicer default font than Computer Modern for most use cases
    \usepackage{palatino}

    % Basic figure setup, for now with no caption control since it's done
    % automatically by Pandoc (which extracts ![](path) syntax from Markdown).
    \usepackage{graphicx}
    % We will generate all images so they have a width \maxwidth. This means
    % that they will get their normal width if they fit onto the page, but
    % are scaled down if they would overflow the margins.
    \makeatletter
    \def\maxwidth{\ifdim\Gin@nat@width>\linewidth\linewidth
    \else\Gin@nat@width\fi}
    \makeatother
    \let\Oldincludegraphics\includegraphics
    % Set max figure width to be 80% of text width, for now hardcoded.
    \renewcommand{\includegraphics}[1]{\Oldincludegraphics[width=.8\maxwidth]{#1}}
    % Ensure that by default, figures have no caption (until we provide a
    % proper Figure object with a Caption API and a way to capture that
    % in the conversion process - todo).
    \usepackage{caption}
    \DeclareCaptionLabelFormat{nolabel}{}
    \captionsetup{labelformat=nolabel}

    \usepackage{adjustbox} % Used to constrain images to a maximum size 
    \usepackage{xcolor} % Allow colors to be defined
    \usepackage{enumerate} % Needed for markdown enumerations to work
    \usepackage{geometry} % Used to adjust the document margins
    \usepackage{amsmath} % Equations
    \usepackage{amssymb} % Equations
    \usepackage{textcomp} % defines textquotesingle
    % Hack from http://tex.stackexchange.com/a/47451/13684:
    \AtBeginDocument{%
        \def\PYZsq{\textquotesingle}% Upright quotes in Pygmentized code
    }
    \usepackage{upquote} % Upright quotes for verbatim code
    \usepackage{eurosym} % defines \euro
    \usepackage[mathletters]{ucs} % Extended unicode (utf-8) support
    \usepackage[utf8x]{inputenc} % Allow utf-8 characters in the tex document
    \usepackage{fancyvrb} % verbatim replacement that allows latex
    \usepackage{grffile} % extends the file name processing of package graphics 
                         % to support a larger range 
    % The hyperref package gives us a pdf with properly built
    % internal navigation ('pdf bookmarks' for the table of contents,
    % internal cross-reference links, web links for URLs, etc.)
    \usepackage{hyperref}
    \usepackage{longtable} % longtable support required by pandoc >1.10
    \usepackage{booktabs}  % table support for pandoc > 1.12.2
    \usepackage[normalem]{ulem} % ulem is needed to support strikethroughs (\sout)
                                % normalem makes italics be italics, not underlines
    

    
    
    % Colors for the hyperref package
    \definecolor{urlcolor}{rgb}{0,.145,.698}
    \definecolor{linkcolor}{rgb}{.71,0.21,0.01}
    \definecolor{citecolor}{rgb}{.12,.54,.11}

    % ANSI colors
    \definecolor{ansi-black}{HTML}{3E424D}
    \definecolor{ansi-black-intense}{HTML}{282C36}
    \definecolor{ansi-red}{HTML}{E75C58}
    \definecolor{ansi-red-intense}{HTML}{B22B31}
    \definecolor{ansi-green}{HTML}{00A250}
    \definecolor{ansi-green-intense}{HTML}{007427}
    \definecolor{ansi-yellow}{HTML}{DDB62B}
    \definecolor{ansi-yellow-intense}{HTML}{B27D12}
    \definecolor{ansi-blue}{HTML}{208FFB}
    \definecolor{ansi-blue-intense}{HTML}{0065CA}
    \definecolor{ansi-magenta}{HTML}{D160C4}
    \definecolor{ansi-magenta-intense}{HTML}{A03196}
    \definecolor{ansi-cyan}{HTML}{60C6C8}
    \definecolor{ansi-cyan-intense}{HTML}{258F8F}
    \definecolor{ansi-white}{HTML}{C5C1B4}
    \definecolor{ansi-white-intense}{HTML}{A1A6B2}

    % commands and environments needed by pandoc snippets
    % extracted from the output of `pandoc -s`
    \providecommand{\tightlist}{%
      \setlength{\itemsep}{0pt}\setlength{\parskip}{0pt}}
    \DefineVerbatimEnvironment{Highlighting}{Verbatim}{commandchars=\\\{\}}
    % Add ',fontsize=\small' for more characters per line
    \newenvironment{Shaded}{}{}
    \newcommand{\KeywordTok}[1]{\textcolor[rgb]{0.00,0.44,0.13}{\textbf{{#1}}}}
    \newcommand{\DataTypeTok}[1]{\textcolor[rgb]{0.56,0.13,0.00}{{#1}}}
    \newcommand{\DecValTok}[1]{\textcolor[rgb]{0.25,0.63,0.44}{{#1}}}
    \newcommand{\BaseNTok}[1]{\textcolor[rgb]{0.25,0.63,0.44}{{#1}}}
    \newcommand{\FloatTok}[1]{\textcolor[rgb]{0.25,0.63,0.44}{{#1}}}
    \newcommand{\CharTok}[1]{\textcolor[rgb]{0.25,0.44,0.63}{{#1}}}
    \newcommand{\StringTok}[1]{\textcolor[rgb]{0.25,0.44,0.63}{{#1}}}
    \newcommand{\CommentTok}[1]{\textcolor[rgb]{0.38,0.63,0.69}{\textit{{#1}}}}
    \newcommand{\OtherTok}[1]{\textcolor[rgb]{0.00,0.44,0.13}{{#1}}}
    \newcommand{\AlertTok}[1]{\textcolor[rgb]{1.00,0.00,0.00}{\textbf{{#1}}}}
    \newcommand{\FunctionTok}[1]{\textcolor[rgb]{0.02,0.16,0.49}{{#1}}}
    \newcommand{\RegionMarkerTok}[1]{{#1}}
    \newcommand{\ErrorTok}[1]{\textcolor[rgb]{1.00,0.00,0.00}{\textbf{{#1}}}}
    \newcommand{\NormalTok}[1]{{#1}}
    
    % Additional commands for more recent versions of Pandoc
    \newcommand{\ConstantTok}[1]{\textcolor[rgb]{0.53,0.00,0.00}{{#1}}}
    \newcommand{\SpecialCharTok}[1]{\textcolor[rgb]{0.25,0.44,0.63}{{#1}}}
    \newcommand{\VerbatimStringTok}[1]{\textcolor[rgb]{0.25,0.44,0.63}{{#1}}}
    \newcommand{\SpecialStringTok}[1]{\textcolor[rgb]{0.73,0.40,0.53}{{#1}}}
    \newcommand{\ImportTok}[1]{{#1}}
    \newcommand{\DocumentationTok}[1]{\textcolor[rgb]{0.73,0.13,0.13}{\textit{{#1}}}}
    \newcommand{\AnnotationTok}[1]{\textcolor[rgb]{0.38,0.63,0.69}{\textbf{\textit{{#1}}}}}
    \newcommand{\CommentVarTok}[1]{\textcolor[rgb]{0.38,0.63,0.69}{\textbf{\textit{{#1}}}}}
    \newcommand{\VariableTok}[1]{\textcolor[rgb]{0.10,0.09,0.49}{{#1}}}
    \newcommand{\ControlFlowTok}[1]{\textcolor[rgb]{0.00,0.44,0.13}{\textbf{{#1}}}}
    \newcommand{\OperatorTok}[1]{\textcolor[rgb]{0.40,0.40,0.40}{{#1}}}
    \newcommand{\BuiltInTok}[1]{{#1}}
    \newcommand{\ExtensionTok}[1]{{#1}}
    \newcommand{\PreprocessorTok}[1]{\textcolor[rgb]{0.74,0.48,0.00}{{#1}}}
    \newcommand{\AttributeTok}[1]{\textcolor[rgb]{0.49,0.56,0.16}{{#1}}}
    \newcommand{\InformationTok}[1]{\textcolor[rgb]{0.38,0.63,0.69}{\textbf{\textit{{#1}}}}}
    \newcommand{\WarningTok}[1]{\textcolor[rgb]{0.38,0.63,0.69}{\textbf{\textit{{#1}}}}}
    
    
    % Define a nice break command that doesn't care if a line doesn't already
    % exist.
    \def\br{\hspace*{\fill} \\* }
    % Math Jax compatability definitions
    \def\gt{>}
    \def\lt{<}
    % Document parameters
    \title{DispatchDesigns}
    
    
    

    % Pygments definitions
    
\makeatletter
\def\PY@reset{\let\PY@it=\relax \let\PY@bf=\relax%
    \let\PY@ul=\relax \let\PY@tc=\relax%
    \let\PY@bc=\relax \let\PY@ff=\relax}
\def\PY@tok#1{\csname PY@tok@#1\endcsname}
\def\PY@toks#1+{\ifx\relax#1\empty\else%
    \PY@tok{#1}\expandafter\PY@toks\fi}
\def\PY@do#1{\PY@bc{\PY@tc{\PY@ul{%
    \PY@it{\PY@bf{\PY@ff{#1}}}}}}}
\def\PY#1#2{\PY@reset\PY@toks#1+\relax+\PY@do{#2}}

\expandafter\def\csname PY@tok@gd\endcsname{\def\PY@tc##1{\textcolor[rgb]{0.63,0.00,0.00}{##1}}}
\expandafter\def\csname PY@tok@gu\endcsname{\let\PY@bf=\textbf\def\PY@tc##1{\textcolor[rgb]{0.50,0.00,0.50}{##1}}}
\expandafter\def\csname PY@tok@gt\endcsname{\def\PY@tc##1{\textcolor[rgb]{0.00,0.27,0.87}{##1}}}
\expandafter\def\csname PY@tok@gs\endcsname{\let\PY@bf=\textbf}
\expandafter\def\csname PY@tok@gr\endcsname{\def\PY@tc##1{\textcolor[rgb]{1.00,0.00,0.00}{##1}}}
\expandafter\def\csname PY@tok@cm\endcsname{\let\PY@it=\textit\def\PY@tc##1{\textcolor[rgb]{0.25,0.50,0.50}{##1}}}
\expandafter\def\csname PY@tok@vg\endcsname{\def\PY@tc##1{\textcolor[rgb]{0.10,0.09,0.49}{##1}}}
\expandafter\def\csname PY@tok@vi\endcsname{\def\PY@tc##1{\textcolor[rgb]{0.10,0.09,0.49}{##1}}}
\expandafter\def\csname PY@tok@vm\endcsname{\def\PY@tc##1{\textcolor[rgb]{0.10,0.09,0.49}{##1}}}
\expandafter\def\csname PY@tok@mh\endcsname{\def\PY@tc##1{\textcolor[rgb]{0.40,0.40,0.40}{##1}}}
\expandafter\def\csname PY@tok@cs\endcsname{\let\PY@it=\textit\def\PY@tc##1{\textcolor[rgb]{0.25,0.50,0.50}{##1}}}
\expandafter\def\csname PY@tok@ge\endcsname{\let\PY@it=\textit}
\expandafter\def\csname PY@tok@vc\endcsname{\def\PY@tc##1{\textcolor[rgb]{0.10,0.09,0.49}{##1}}}
\expandafter\def\csname PY@tok@il\endcsname{\def\PY@tc##1{\textcolor[rgb]{0.40,0.40,0.40}{##1}}}
\expandafter\def\csname PY@tok@go\endcsname{\def\PY@tc##1{\textcolor[rgb]{0.53,0.53,0.53}{##1}}}
\expandafter\def\csname PY@tok@cp\endcsname{\def\PY@tc##1{\textcolor[rgb]{0.74,0.48,0.00}{##1}}}
\expandafter\def\csname PY@tok@gi\endcsname{\def\PY@tc##1{\textcolor[rgb]{0.00,0.63,0.00}{##1}}}
\expandafter\def\csname PY@tok@gh\endcsname{\let\PY@bf=\textbf\def\PY@tc##1{\textcolor[rgb]{0.00,0.00,0.50}{##1}}}
\expandafter\def\csname PY@tok@ni\endcsname{\let\PY@bf=\textbf\def\PY@tc##1{\textcolor[rgb]{0.60,0.60,0.60}{##1}}}
\expandafter\def\csname PY@tok@nl\endcsname{\def\PY@tc##1{\textcolor[rgb]{0.63,0.63,0.00}{##1}}}
\expandafter\def\csname PY@tok@nn\endcsname{\let\PY@bf=\textbf\def\PY@tc##1{\textcolor[rgb]{0.00,0.00,1.00}{##1}}}
\expandafter\def\csname PY@tok@no\endcsname{\def\PY@tc##1{\textcolor[rgb]{0.53,0.00,0.00}{##1}}}
\expandafter\def\csname PY@tok@na\endcsname{\def\PY@tc##1{\textcolor[rgb]{0.49,0.56,0.16}{##1}}}
\expandafter\def\csname PY@tok@nb\endcsname{\def\PY@tc##1{\textcolor[rgb]{0.00,0.50,0.00}{##1}}}
\expandafter\def\csname PY@tok@nc\endcsname{\let\PY@bf=\textbf\def\PY@tc##1{\textcolor[rgb]{0.00,0.00,1.00}{##1}}}
\expandafter\def\csname PY@tok@nd\endcsname{\def\PY@tc##1{\textcolor[rgb]{0.67,0.13,1.00}{##1}}}
\expandafter\def\csname PY@tok@ne\endcsname{\let\PY@bf=\textbf\def\PY@tc##1{\textcolor[rgb]{0.82,0.25,0.23}{##1}}}
\expandafter\def\csname PY@tok@nf\endcsname{\def\PY@tc##1{\textcolor[rgb]{0.00,0.00,1.00}{##1}}}
\expandafter\def\csname PY@tok@si\endcsname{\let\PY@bf=\textbf\def\PY@tc##1{\textcolor[rgb]{0.73,0.40,0.53}{##1}}}
\expandafter\def\csname PY@tok@s2\endcsname{\def\PY@tc##1{\textcolor[rgb]{0.73,0.13,0.13}{##1}}}
\expandafter\def\csname PY@tok@nt\endcsname{\let\PY@bf=\textbf\def\PY@tc##1{\textcolor[rgb]{0.00,0.50,0.00}{##1}}}
\expandafter\def\csname PY@tok@nv\endcsname{\def\PY@tc##1{\textcolor[rgb]{0.10,0.09,0.49}{##1}}}
\expandafter\def\csname PY@tok@s1\endcsname{\def\PY@tc##1{\textcolor[rgb]{0.73,0.13,0.13}{##1}}}
\expandafter\def\csname PY@tok@dl\endcsname{\def\PY@tc##1{\textcolor[rgb]{0.73,0.13,0.13}{##1}}}
\expandafter\def\csname PY@tok@ch\endcsname{\let\PY@it=\textit\def\PY@tc##1{\textcolor[rgb]{0.25,0.50,0.50}{##1}}}
\expandafter\def\csname PY@tok@m\endcsname{\def\PY@tc##1{\textcolor[rgb]{0.40,0.40,0.40}{##1}}}
\expandafter\def\csname PY@tok@gp\endcsname{\let\PY@bf=\textbf\def\PY@tc##1{\textcolor[rgb]{0.00,0.00,0.50}{##1}}}
\expandafter\def\csname PY@tok@sh\endcsname{\def\PY@tc##1{\textcolor[rgb]{0.73,0.13,0.13}{##1}}}
\expandafter\def\csname PY@tok@ow\endcsname{\let\PY@bf=\textbf\def\PY@tc##1{\textcolor[rgb]{0.67,0.13,1.00}{##1}}}
\expandafter\def\csname PY@tok@sx\endcsname{\def\PY@tc##1{\textcolor[rgb]{0.00,0.50,0.00}{##1}}}
\expandafter\def\csname PY@tok@bp\endcsname{\def\PY@tc##1{\textcolor[rgb]{0.00,0.50,0.00}{##1}}}
\expandafter\def\csname PY@tok@c1\endcsname{\let\PY@it=\textit\def\PY@tc##1{\textcolor[rgb]{0.25,0.50,0.50}{##1}}}
\expandafter\def\csname PY@tok@fm\endcsname{\def\PY@tc##1{\textcolor[rgb]{0.00,0.00,1.00}{##1}}}
\expandafter\def\csname PY@tok@o\endcsname{\def\PY@tc##1{\textcolor[rgb]{0.40,0.40,0.40}{##1}}}
\expandafter\def\csname PY@tok@kc\endcsname{\let\PY@bf=\textbf\def\PY@tc##1{\textcolor[rgb]{0.00,0.50,0.00}{##1}}}
\expandafter\def\csname PY@tok@c\endcsname{\let\PY@it=\textit\def\PY@tc##1{\textcolor[rgb]{0.25,0.50,0.50}{##1}}}
\expandafter\def\csname PY@tok@mf\endcsname{\def\PY@tc##1{\textcolor[rgb]{0.40,0.40,0.40}{##1}}}
\expandafter\def\csname PY@tok@err\endcsname{\def\PY@bc##1{\setlength{\fboxsep}{0pt}\fcolorbox[rgb]{1.00,0.00,0.00}{1,1,1}{\strut ##1}}}
\expandafter\def\csname PY@tok@mb\endcsname{\def\PY@tc##1{\textcolor[rgb]{0.40,0.40,0.40}{##1}}}
\expandafter\def\csname PY@tok@ss\endcsname{\def\PY@tc##1{\textcolor[rgb]{0.10,0.09,0.49}{##1}}}
\expandafter\def\csname PY@tok@sr\endcsname{\def\PY@tc##1{\textcolor[rgb]{0.73,0.40,0.53}{##1}}}
\expandafter\def\csname PY@tok@mo\endcsname{\def\PY@tc##1{\textcolor[rgb]{0.40,0.40,0.40}{##1}}}
\expandafter\def\csname PY@tok@kd\endcsname{\let\PY@bf=\textbf\def\PY@tc##1{\textcolor[rgb]{0.00,0.50,0.00}{##1}}}
\expandafter\def\csname PY@tok@mi\endcsname{\def\PY@tc##1{\textcolor[rgb]{0.40,0.40,0.40}{##1}}}
\expandafter\def\csname PY@tok@kn\endcsname{\let\PY@bf=\textbf\def\PY@tc##1{\textcolor[rgb]{0.00,0.50,0.00}{##1}}}
\expandafter\def\csname PY@tok@cpf\endcsname{\let\PY@it=\textit\def\PY@tc##1{\textcolor[rgb]{0.25,0.50,0.50}{##1}}}
\expandafter\def\csname PY@tok@kr\endcsname{\let\PY@bf=\textbf\def\PY@tc##1{\textcolor[rgb]{0.00,0.50,0.00}{##1}}}
\expandafter\def\csname PY@tok@s\endcsname{\def\PY@tc##1{\textcolor[rgb]{0.73,0.13,0.13}{##1}}}
\expandafter\def\csname PY@tok@kp\endcsname{\def\PY@tc##1{\textcolor[rgb]{0.00,0.50,0.00}{##1}}}
\expandafter\def\csname PY@tok@w\endcsname{\def\PY@tc##1{\textcolor[rgb]{0.73,0.73,0.73}{##1}}}
\expandafter\def\csname PY@tok@kt\endcsname{\def\PY@tc##1{\textcolor[rgb]{0.69,0.00,0.25}{##1}}}
\expandafter\def\csname PY@tok@sc\endcsname{\def\PY@tc##1{\textcolor[rgb]{0.73,0.13,0.13}{##1}}}
\expandafter\def\csname PY@tok@sb\endcsname{\def\PY@tc##1{\textcolor[rgb]{0.73,0.13,0.13}{##1}}}
\expandafter\def\csname PY@tok@sa\endcsname{\def\PY@tc##1{\textcolor[rgb]{0.73,0.13,0.13}{##1}}}
\expandafter\def\csname PY@tok@k\endcsname{\let\PY@bf=\textbf\def\PY@tc##1{\textcolor[rgb]{0.00,0.50,0.00}{##1}}}
\expandafter\def\csname PY@tok@se\endcsname{\let\PY@bf=\textbf\def\PY@tc##1{\textcolor[rgb]{0.73,0.40,0.13}{##1}}}
\expandafter\def\csname PY@tok@sd\endcsname{\let\PY@it=\textit\def\PY@tc##1{\textcolor[rgb]{0.73,0.13,0.13}{##1}}}

\def\PYZbs{\char`\\}
\def\PYZus{\char`\_}
\def\PYZob{\char`\{}
\def\PYZcb{\char`\}}
\def\PYZca{\char`\^}
\def\PYZam{\char`\&}
\def\PYZlt{\char`\<}
\def\PYZgt{\char`\>}
\def\PYZsh{\char`\#}
\def\PYZpc{\char`\%}
\def\PYZdl{\char`\$}
\def\PYZhy{\char`\-}
\def\PYZsq{\char`\'}
\def\PYZdq{\char`\"}
\def\PYZti{\char`\~}
% for compatibility with earlier versions
\def\PYZat{@}
\def\PYZlb{[}
\def\PYZrb{]}
\makeatother


    % Exact colors from NB
    \definecolor{incolor}{rgb}{0.0, 0.0, 0.5}
    \definecolor{outcolor}{rgb}{0.545, 0.0, 0.0}



    
    % Prevent overflowing lines due to hard-to-break entities
    \sloppy 
    % Setup hyperref package
    \hypersetup{
      breaklinks=true,  % so long urls are correctly broken across lines
      colorlinks=true,
      urlcolor=urlcolor,
      linkcolor=linkcolor,
      citecolor=citecolor,
      }
    % Slightly bigger margins than the latex defaults
    
    \geometry{verbose,tmargin=1in,bmargin=1in,lmargin=1in,rmargin=1in}
    
    

    \begin{document}
    
    
    \maketitle
    
    

    
    \section{Multiple Dispatch Designs: Duck Typing, Hierarchies and
Traits}\label{multiple-dispatch-designs-duck-typing-hierarchies-and-traits}

\subsection{Introduction to Dispatch
Designs}\label{introduction-to-dispatch-designs}

Julia is built around types. Software architectures in Julia are built
around good use of the type system. This makes it easy to build generic
code which works over a large range of types and gets good performance.
The result is high-performance code that has many features. In fact,
with generic typing, your code may have more features than you know of!
The purpose of this tutorial is to introduce the multiple dispatch
designs that allow this to happen.

    \subsection{Duck Typing}\label{duck-typing}

If it quacks like a duck, it might as well be a duck. This is the idea
of defining an object by the way that it acts. This idea is central to
type-based designs: \textbf{abstract types are defined by how they act}.
For example, a \texttt{Number} is some type that can do things like
\texttt{+},\texttt{-},\texttt{*}, and \texttt{/}. In this category we
have things like \texttt{Float64} and \texttt{Int32}. An
\texttt{AbstractFloat} is some floating point number, and so it should
have a dispatch of \texttt{eps(T)} that gives its machine epsilon. An
\texttt{AbstractArray} is a type that can be indexed like
\texttt{A{[}i{]}}. An \texttt{AbstractArray} may be mutable, meaning it
can be ``set'': \texttt{A{[}i{]}=v}.

These abstract types then have actions which abstract from their
underlying implmentation. \texttt{A.*B} does element-wise
multiplication, and in many cases it does not matter what kind of array
this is done on. The default is \texttt{Array} which is a contiguous
array on the CPU, but this action is common amongst
\texttt{AbstractArray} types. If a user has a \texttt{DistributedArray}
(\texttt{DArray}), then \texttt{A.*B} will work on multiple nodes of a
cluster. If the user uses a \texttt{GPUArray}, then \texttt{A.*B} will
be performed on the GPU. Thus, if you don't restrict the usage of your
algorithm to \texttt{Array}, then your algorithm actually is ``just
works'' as many different algorithms.

This is all well and good, but this would not be worthwhile if it was
not performant. Thankfully, Julia has an answer to this. Every function
auto-specializes on the types which it is given. Thus if you look at
something like:

    \begin{Verbatim}[commandchars=\\\{\}]
{\color{incolor}In [{\color{incolor}1}]:} \PY{n}{my\PYZus{}square}\PY{p}{(}\PY{n}{x}\PY{p}{)} \PY{o}{=} \PY{n}{x}\PY{o}{\PYZca{}}\PY{l+m+mi}{2}
\end{Verbatim}

            \begin{Verbatim}[commandchars=\\\{\}]
{\color{outcolor}Out[{\color{outcolor}1}]:} my\_square (generic function with 1 method)
\end{Verbatim}
        
    then we see that this function will be efficient for the types that we
give it. Looking at the generated code:

    \begin{Verbatim}[commandchars=\\\{\}]
{\color{incolor}In [{\color{incolor}2}]:} \PY{n+nd}{@code\PYZus{}llvm} \PY{n}{my\PYZus{}square}\PY{p}{(}\PY{l+m+mi}{1}\PY{p}{)}
\end{Verbatim}

    \begin{Verbatim}[commandchars=\\\{\}]

define i64 @julia\_my\_square\_72669(i64) \#0 \{
top:
  \%1 = mul i64 \%0, \%0
  ret i64 \%1
\}

    \end{Verbatim}

    \begin{Verbatim}[commandchars=\\\{\}]
{\color{incolor}In [{\color{incolor}3}]:} \PY{n+nd}{@code\PYZus{}llvm} \PY{n}{my\PYZus{}square}\PY{p}{(}\PY{l+m+mf}{1.0}\PY{p}{)}
\end{Verbatim}

    \begin{Verbatim}[commandchars=\\\{\}]

define double @julia\_my\_square\_72684(double) \#0 \{
top:
  \%1 = fmul double \%0, \%0
  ret double \%1
\}

    \end{Verbatim}

    See that the function which is generated by the compiler is different in
each case. The first specifically is an integer multiplication
\texttt{x*x} of the input \texttt{x}. The other is a floating point
multiplication \texttt{x*x} of the input \texttt{x}. But this means that
it does not matter what kind of \texttt{Number} we put in here: this
function will work as long as \texttt{*} is defined, and it will be
efficient by Julia's multiple dispatch design.

Thus we don't need to restrict the type in order to get performance.
That means that

    \begin{Verbatim}[commandchars=\\\{\}]
{\color{incolor}In [{\color{incolor}4}]:} \PY{n}{my\PYZus{}restricted\PYZus{}square}\PY{p}{(}\PY{n}{x}\PY{o}{::}\PY{k+kt}{Int}\PY{p}{)} \PY{o}{=} \PY{n}{x}\PY{o}{\PYZca{}}\PY{l+m+mi}{2}
\end{Verbatim}

            \begin{Verbatim}[commandchars=\\\{\}]
{\color{outcolor}Out[{\color{outcolor}4}]:} my\_restricted\_square (generic function with 1 method)
\end{Verbatim}
        
    is no more efficient than the version above, and actually generates the
same exact compiled code:

    \begin{Verbatim}[commandchars=\\\{\}]
{\color{incolor}In [{\color{incolor}5}]:} \PY{n+nd}{@code\PYZus{}llvm} \PY{n}{my\PYZus{}restricted\PYZus{}square}\PY{p}{(}\PY{l+m+mi}{1}\PY{p}{)}
\end{Verbatim}

    \begin{Verbatim}[commandchars=\\\{\}]

define i64 @julia\_my\_restricted\_square\_72686(i64) \#0 \{
top:
  \%1 = mul i64 \%0, \%0
  ret i64 \%1
\}

    \end{Verbatim}

    Thus we can write generic and efficient code by leaving our functions
unrestricted. This is the practice of duck-typing functions. We just let
them work on any input types. If the type has the correct actions, the
function will ``just work''. If it does not have the correct actions,
for our example above say \texttt{*} is undefined, then a
\texttt{MethodError} saying the action is not defined will be thrown.

We can be slightly more conservative by restricting to abstract types.
For example:

    \begin{Verbatim}[commandchars=\\\{\}]
{\color{incolor}In [{\color{incolor}6}]:} \PY{n}{my\PYZus{}number\PYZus{}restricted\PYZus{}square}\PY{p}{(}\PY{n}{x}\PY{o}{::}\PY{k+kt}{Number}\PY{p}{)} \PY{o}{=} \PY{n}{x}\PY{o}{\PYZca{}}\PY{l+m+mi}{2}
\end{Verbatim}

            \begin{Verbatim}[commandchars=\\\{\}]
{\color{outcolor}Out[{\color{outcolor}6}]:} my\_number\_restricted\_square (generic function with 1 method)
\end{Verbatim}
        
    will allow any \texttt{Number}. There are things which can square which
aren't \texttt{Number}s for which this will now throw an error (a matrix
is a simple example). But, this can let us clearly define the interface
for our package/script/code. Using these assertions, we can then
dispatch differently for different type classes. For example:

    \begin{Verbatim}[commandchars=\\\{\}]
{\color{incolor}In [{\color{incolor}7}]:} \PY{n}{my\PYZus{}number\PYZus{}restricted\PYZus{}square}\PY{p}{(}\PY{n}{x}\PY{o}{::}\PY{k+kt}{AbstractArray}\PY{p}{)} \PY{o}{=} \PY{p}{(}\PY{n}{println}\PY{p}{(}\PY{n}{x}\PY{p}{)}\PY{p}{;}\PY{n}{x}\PY{o}{.\PYZca{}}\PY{l+m+mi}{2}\PY{p}{)}
\end{Verbatim}

            \begin{Verbatim}[commandchars=\\\{\}]
{\color{outcolor}Out[{\color{outcolor}7}]:} my\_number\_restricted\_square (generic function with 2 methods)
\end{Verbatim}
        
    Now, \texttt{my\_number\_restricted\_square} calculates \texttt{x\^{}2}
on a \texttt{Number}, and for an array it will print the array and
calculate \texttt{x\^{}2} element-wise. Thus we are controlling behavior
with broad strokes using classes of types and their associated actions.

    \subsection{Type Hierarchies}\label{type-hierarchies}

This idea of control leads to type hierarchies. In object-oriented
programming languages, you sort objects by their implementation. Fields,
the pieces of data that an object holds, are what is inherited.

There is an inherent limitation to that kind of thinking when looking to
achieve good performance. In many cases, you don't need as much data to
do an action. A good example of this is the range type, for example
\texttt{1:10}.

    \begin{Verbatim}[commandchars=\\\{\}]
{\color{incolor}In [{\color{incolor}8}]:} \PY{n}{a} \PY{o}{=} \PY{l+m+mi}{1}\PY{o}{:}\PY{l+m+mi}{10}
\end{Verbatim}

            \begin{Verbatim}[commandchars=\\\{\}]
{\color{outcolor}Out[{\color{outcolor}8}]:} 1:10
\end{Verbatim}
        
    This type is an abstract array:

    \begin{Verbatim}[commandchars=\\\{\}]
{\color{incolor}In [{\color{incolor}9}]:} \PY{n}{typeof}\PY{p}{(}\PY{n}{a}\PY{p}{)} \PY{o}{\PYZlt{}:} \PY{k+kt}{AbstractArray}
\end{Verbatim}

            \begin{Verbatim}[commandchars=\\\{\}]
{\color{outcolor}Out[{\color{outcolor}9}]:} true
\end{Verbatim}
        
    It has actions like an \texttt{Array}

    \begin{Verbatim}[commandchars=\\\{\}]
{\color{incolor}In [{\color{incolor}10}]:} \PY{n}{a}\PY{p}{[}\PY{l+m+mi}{2}\PY{p}{]}
\end{Verbatim}

            \begin{Verbatim}[commandchars=\\\{\}]
{\color{outcolor}Out[{\color{outcolor}10}]:} 2
\end{Verbatim}
        
    However, it is not an \texttt{Array}. In fact, it's just two numbers. We
can see this by looking at its fields:

    \begin{Verbatim}[commandchars=\\\{\}]
{\color{incolor}In [{\color{incolor}11}]:} \PY{n}{fieldnames}\PY{p}{(}\PY{n}{a}\PY{p}{)}
\end{Verbatim}

            \begin{Verbatim}[commandchars=\\\{\}]
{\color{outcolor}Out[{\color{outcolor}11}]:} 2-element Array\{Symbol,1\}:
          :start
          :stop 
\end{Verbatim}
        
    It is an \texttt{immutable} type which just holds the start and stop
values. This means that its indexing, \texttt{A{[}i{]}}, is just a
function. What's nice about this is that means that no array is ever
created. Creating large arrays can be a costly action:

    \begin{Verbatim}[commandchars=\\\{\}]
{\color{incolor}In [{\color{incolor}12}]:} \PY{n+nd}{@time} \PY{n}{collect}\PY{p}{(}\PY{l+m+mi}{1}\PY{o}{:}\PY{l+m+mi}{10000000}\PY{p}{)}
\end{Verbatim}

    \begin{Verbatim}[commandchars=\\\{\}]
  0.038615 seconds (308 allocations: 76.312 MB, 45.16\% gc time)

    \end{Verbatim}

    But creating an immutable type of two numbers is essentially free, no
matter what those two numbers are:

    \begin{Verbatim}[commandchars=\\\{\}]
{\color{incolor}In [{\color{incolor}13}]:} \PY{n+nd}{@time} \PY{l+m+mi}{1}\PY{o}{:}\PY{l+m+mi}{10000000}
\end{Verbatim}

    \begin{Verbatim}[commandchars=\\\{\}]
  0.000001 seconds (5 allocations: 192 bytes)

    \end{Verbatim}

            \begin{Verbatim}[commandchars=\\\{\}]
{\color{outcolor}Out[{\color{outcolor}13}]:} 1:10000000
\end{Verbatim}
        
    The array takes $\mathcal{O}(n)$ memory to store its values while this
type is $\mathcal{O}(1)$, using a constant 192 bytes (if the start and
stop are \texttt{Int64}). Yet, in cases where we just want to index
values, they act exactly the same.

Another nice example is the \texttt{UniformScaling} operator, which acts
like an identity matrix without forming an identity matrix.

    \begin{Verbatim}[commandchars=\\\{\}]
{\color{incolor}In [{\color{incolor}14}]:} \PY{n}{println}\PY{p}{(}\PY{n+nb}{I}\PY{p}{[}\PY{l+m+mi}{10}\PY{p}{,}\PY{l+m+mi}{10}\PY{p}{]}\PY{p}{)}
         \PY{n}{println}\PY{p}{(}\PY{n+nb}{I}\PY{p}{[}\PY{l+m+mi}{10}\PY{p}{,}\PY{l+m+mi}{2}\PY{p}{]}\PY{p}{)}
\end{Verbatim}

    \begin{Verbatim}[commandchars=\\\{\}]
1
0

    \end{Verbatim}

    This can calculate expressions like \texttt{A-b*I} without ever forming
the matrix \texttt{eye(n)} which would take $\mathcal{O}(n^2)$ memory.

    This means that a lot of efficiency can be gained by generalizing our
algorithms to allow for generic typing and organization around actions.
This means that, while in an object-oriented programming language you
group by implementation details, in typed-dispatch programming you group
by actions. \texttt{Number} is an abstract type for ``things which act
like numbers, i.e.~do things like \texttt{*}'', while
\texttt{AbstractArray} is for ``things which index and sometimes set''.

This is the key idea to keep in mind when building type hierarchies:
things which subtype are inheriting behavior. You should setup your
\texttt{abstract} types to mean the existance or non-existance of some
behavior. For example:

    \begin{Verbatim}[commandchars=\\\{\}]
{\color{incolor}In [{\color{incolor}15}]:} \PY{k}{abstract} \PY{k}{type} \PY{n}{AbstractPerson} \PY{k}{end}
         \PY{k}{abstract} \PY{k}{type} \PY{n}{AbstractStudent} \PY{o}{\PYZlt{}:} \PY{n}{AbstractPerson} \PY{k}{end}
         \PY{k}{abstract} \PY{k}{type} \PY{n}{AbstractTeacher} \PY{o}{\PYZlt{}:} \PY{n}{AbstractPerson} \PY{k}{end}
         
         \PY{n}{mutable} \PY{n}{struct} \PY{n}{Person} \PY{o}{\PYZlt{}:} \PY{n}{AbstractPerson}
           \PY{n}{name}\PY{o}{::}\PY{n}{String}    
         \PY{k}{end}
         
         \PY{n}{mutable} \PY{n}{struct} \PY{n}{Student} \PY{o}{\PYZlt{}:} \PY{n}{AbstractStudent}
           \PY{n}{name}\PY{o}{::}\PY{n}{String}  
           \PY{n}{grade}\PY{o}{::}\PY{k+kt}{Int}
           \PY{n}{hobby}\PY{o}{::}\PY{n}{String}
         \PY{k}{end}
         
         \PY{n}{mutable} \PY{n}{struct} \PY{n}{MusicStudent} \PY{o}{\PYZlt{}:} \PY{n}{AbstractStudent}
           \PY{n}{grade}\PY{o}{::}\PY{k+kt}{Int}
         \PY{k}{end}
         
         \PY{n}{mutable} \PY{n}{struct} \PY{n}{Teacher} \PY{o}{\PYZlt{}:} \PY{n}{AbstractTeacher}
           \PY{n}{name}\PY{o}{::}\PY{n}{String}
           \PY{n}{grade}\PY{o}{::}\PY{k+kt}{Int}
         \PY{k}{end}
\end{Verbatim}

    This can be interpreted as follows. At the top we have
\texttt{AbstractPerson}. Our interface here is ``a Person is someone who
has a name which can be gotten by \texttt{get\_name}''.

    \begin{Verbatim}[commandchars=\\\{\}]
{\color{incolor}In [{\color{incolor}16}]:} \PY{n}{get\PYZus{}name}\PY{p}{(}\PY{n}{x}\PY{o}{::}\PY{n}{AbstractPerson}\PY{p}{)} \PY{o}{=} \PY{n}{x}\PY{o}{.}\PY{n}{name}
\end{Verbatim}

            \begin{Verbatim}[commandchars=\\\{\}]
{\color{outcolor}Out[{\color{outcolor}16}]:} get\_name (generic function with 1 method)
\end{Verbatim}
        
    Thus codes which are written for an \texttt{AbstractPerson} can ``know''
(by our informal declaration of the interface) that \texttt{get\_name}
will ``just work'' for its subtypes. However, notice that
\texttt{MusicStudent} doesn't have a \texttt{name} field. This is
because \texttt{MusicStudent}s just want to be named whatever the
trendiest band is, so we can just replace the usage of the field by the
action:

    \begin{Verbatim}[commandchars=\\\{\}]
{\color{incolor}In [{\color{incolor}17}]:} \PY{n}{get\PYZus{}name}\PY{p}{(}\PY{n}{x}\PY{o}{::}\PY{n}{MusicStudent}\PY{p}{)} \PY{o}{=} \PY{l+s}{\PYZdq{}}\PY{l+s}{J}\PY{l+s}{u}\PY{l+s}{s}\PY{l+s}{t}\PY{l+s}{i}\PY{l+s}{n}\PY{l+s}{ }\PY{l+s}{B}\PY{l+s}{i}\PY{l+s}{e}\PY{l+s}{b}\PY{l+s}{e}\PY{l+s}{r}\PY{l+s}{\PYZdq{}}
\end{Verbatim}

            \begin{Verbatim}[commandchars=\\\{\}]
{\color{outcolor}Out[{\color{outcolor}17}]:} get\_name (generic function with 2 methods)
\end{Verbatim}
        
    In this way, we can use \texttt{get\_name} to get the name, and how it
was implemented (whether it's pulling something that had to be stored
from memory, or if it's something magically known in advance) does not
matter. We can keep refining this: an \texttt{AbstractStudent} has a
\texttt{get\_hobby}, but a \texttt{MusicStudent}'s hobby is always
\texttt{Music}, so there's not reason to store that data in the type and
instead just have its actions implicitly ``know'' this. In non-trivial
examples (like the range and \texttt{UniformScaling} above), this
distinction by action and abstraction away from the actual
implementation of the types allows for full optimization of generic
codes.

    \subsection{Small Functions and Constant
Propagation}\label{small-functions-and-constant-propagation}

    The next question to ask is, does storing information in functions and
actions affect performance? The answer is yes, and in favor of the
function approach! To see this, let's see what happens when we use these
functions. To make it simpler, let's use a boolean function. Teachers
are old and don't like music, while students do like music. But
generally people like music. This means that:

    \begin{Verbatim}[commandchars=\\\{\}]
{\color{incolor}In [{\color{incolor}18}]:} \PY{n}{likes\PYZus{}music}\PY{p}{(}\PY{n}{x}\PY{o}{::}\PY{n}{AbstractTeacher}\PY{p}{)} \PY{o}{=} \PY{k+kc}{false}
         \PY{n}{likes\PYZus{}music}\PY{p}{(}\PY{n}{x}\PY{o}{::}\PY{n}{AbstractStudent}\PY{p}{)} \PY{o}{=} \PY{k+kc}{true}
         \PY{n}{likes\PYZus{}music}\PY{p}{(}\PY{n}{x}\PY{o}{::}\PY{n}{AbstractPerson}\PY{p}{)} \PY{o}{=} \PY{k+kc}{true}
\end{Verbatim}

            \begin{Verbatim}[commandchars=\\\{\}]
{\color{outcolor}Out[{\color{outcolor}18}]:} likes\_music (generic function with 3 methods)
\end{Verbatim}
        
    Now how many records would these people buy at a record store? If they
don't like music, they will buy zero records. If they like music, then
they will pick up a random number between 1 and 10. If they are a
student, they will then double that (impulsive Millenials!).

    \begin{Verbatim}[commandchars=\\\{\}]
{\color{incolor}In [{\color{incolor}19}]:} \PY{k}{function} \PY{n}{number\PYZus{}of\PYZus{}records}\PY{p}{(}\PY{n}{x}\PY{o}{::}\PY{n}{AbstractPerson}\PY{p}{)}
             \PY{k}{if} \PY{o}{!}\PY{n}{likes\PYZus{}music}\PY{p}{(}\PY{n}{x}\PY{p}{)} 
               \PY{k}{return} \PY{l+m+mi}{0}
             \PY{k}{end}
             \PY{n}{num\PYZus{}records} \PY{o}{=} \PY{n}{rand}\PY{p}{(}\PY{l+m+mi}{10}\PY{p}{)}
             \PY{k}{if} \PY{n}{typeof}\PY{p}{(}\PY{n}{x}\PY{p}{)} \PY{o}{\PYZlt{}:} \PY{n}{AbstractStudent}
               \PY{k}{return} \PY{l+m+mi}{2}\PY{n}{num\PYZus{}records}
             \PY{k}{else} 
               \PY{k}{return} \PY{n}{num\PYZus{}records}
             \PY{k}{end}
         \PY{k}{end}
\end{Verbatim}

            \begin{Verbatim}[commandchars=\\\{\}]
{\color{outcolor}Out[{\color{outcolor}19}]:} number\_of\_records (generic function with 1 method)
\end{Verbatim}
        
    Let's check the code that is created:

    \begin{Verbatim}[commandchars=\\\{\}]
{\color{incolor}In [{\color{incolor} }]:} \PY{n}{x} \PY{o}{=} \PY{n}{Teacher}\PY{p}{(}\PY{l+s}{\PYZdq{}}\PY{l+s}{R}\PY{l+s}{a}\PY{l+s}{n}\PY{l+s}{d}\PY{l+s}{y}\PY{l+s}{\PYZdq{}}\PY{p}{,}\PY{l+m+mi}{11}\PY{p}{)}
        \PY{n}{println}\PY{p}{(}\PY{n}{number\PYZus{}of\PYZus{}records}\PY{p}{(}\PY{n}{x}\PY{p}{)}\PY{p}{)}
        \PY{n+nd}{@code\PYZus{}llvm} \PY{n}{number\PYZus{}of\PYZus{}records}\PY{p}{(}\PY{n}{x}\PY{p}{)}
\end{Verbatim}

    on v0.6, we get:

\begin{Shaded}
\begin{Highlighting}[]
\NormalTok{; }\DataTypeTok{Function} \NormalTok{Attrs: uwtable}
\NormalTok{define i64 @julia_number_of_records_}\FloatTok{63848}\NormalTok{(i8** dereferenceable(}\FloatTok{16}\NormalTok{)) }\CommentTok{#0 !dbg !5 \{}
\NormalTok{top:}
  \NormalTok{ret i64 }\FloatTok{0}
\NormalTok{\}}
\end{Highlighting}
\end{Shaded}

Notice that the entire function compiled away! Then for a music student

\begin{Shaded}
\begin{Highlighting}[]
\NormalTok{x = MusicStudent(}\FloatTok{10}\NormalTok{)}
\NormalTok{@code_typed number_of_records(x)}

\CommentTok{# Output}
\NormalTok{CodeInfo(:(}\KeywordTok{begin} 
        \NormalTok{NewvarNode(:(num_records))}
        \NormalTok{goto }\FloatTok{4} \CommentTok{# line 30:}
        \FloatTok{4}\NormalTok{:  }\CommentTok{# line 32:}
        \NormalTok{$(}\DataTypeTok{Expr}\NormalTok{(:inbounds, false))}
        \CommentTok{# meta: location random.jl rand 279}
        \CommentTok{# meta: location random.jl rand 278}
        \CommentTok{# meta: location random.jl rand 367}
        \CommentTok{# meta: location random.jl rand 370}
        \NormalTok{SSAValue(}\FloatTok{1}\NormalTok{) = $(}\DataTypeTok{Expr}\NormalTok{(:foreigncall, :(:jl_alloc_array_}\FloatTok{1}\NormalTok{d), }\DataTypeTok{Array}\NormalTok{\{}\DataTypeTok{Float64}\NormalTok{,}\FloatTok{1}\NormalTok{\}, svec(}\DataTypeTok{Any}\NormalTok{, }\DataTypeTok{Int64}\NormalTok{), }\DataTypeTok{Array}\NormalTok{\{}\DataTypeTok{Float64}\NormalTok{,}\FloatTok{1}\NormalTok{\}, }\FloatTok{0}\NormalTok{, }\FloatTok{10}\NormalTok{, }\FloatTok{0}\NormalTok{))}
        \CommentTok{# meta: pop location}
        \CommentTok{# meta: pop location}
        \CommentTok{# meta: pop location}
        \CommentTok{# meta: pop location}
        \NormalTok{$(}\DataTypeTok{Expr}\NormalTok{(:inbounds, :pop))}
        \NormalTok{num_records = $(}\DataTypeTok{Expr}\NormalTok{(:invoke, MethodInstance }\KeywordTok{for} \NormalTok{rand!(::MersenneTwister, ::}\DataTypeTok{Array}\NormalTok{\{}\DataTypeTok{Float64}\NormalTok{,}\FloatTok{1}\NormalTok{\}, ::}\DataTypeTok{Int64}\NormalTok{, ::}\DataTypeTok{Type}\NormalTok{\{Base.Random.CloseOpen\}), :(Base.Random.rand!), :(Base.Random.GLOBAL_RNG), SSAValue(}\FloatTok{1}\NormalTok{), :((Base.arraylen)(SSAValue(}\FloatTok{1}\NormalTok{))::}\DataTypeTok{Int64}\NormalTok{), :(Base.Random.CloseOpen))) }\CommentTok{# line 33:}
        \NormalTok{(MusicStudent <: Main.AbstractStudent)::}\DataTypeTok{Bool} \CommentTok{# line 34:}
        \KeywordTok{return} \NormalTok{$(}\DataTypeTok{Expr}\NormalTok{(:invoke, MethodInstance }\KeywordTok{for} \NormalTok{*(::}\DataTypeTok{Int64}\NormalTok{, ::}\DataTypeTok{Array}\NormalTok{\{}\DataTypeTok{Float64}\NormalTok{,}\FloatTok{1}\NormalTok{\}), :(Main.*), }\FloatTok{2}\NormalTok{, :(num_records))) }\CommentTok{# line 36:}
    \KeywordTok{end}\NormalTok{))=>}\DataTypeTok{Array}\NormalTok{\{}\DataTypeTok{Float64}\NormalTok{,}\FloatTok{1}\NormalTok{\}}
\end{Highlighting}
\end{Shaded}

we get a multiplication by 2, while for a regular person,

\begin{Shaded}
\begin{Highlighting}[]
\NormalTok{x = Person(}\StringTok{"Miguel"}\NormalTok{)}
\NormalTok{@code_typed number_of_records(x)}

\CommentTok{# Output}
\NormalTok{CodeInfo(:(}\KeywordTok{begin} 
        \NormalTok{NewvarNode(:(num_records))}
        \NormalTok{goto }\FloatTok{4} \CommentTok{# line 30:}
        \FloatTok{4}\NormalTok{:  }\CommentTok{# line 32:}
        \NormalTok{$(}\DataTypeTok{Expr}\NormalTok{(:inbounds, false))}
        \CommentTok{# meta: location random.jl rand 279}
        \CommentTok{# meta: location random.jl rand 278}
        \CommentTok{# meta: location random.jl rand 367}
        \CommentTok{# meta: location random.jl rand 370}
        \NormalTok{SSAValue(}\FloatTok{1}\NormalTok{) = $(}\DataTypeTok{Expr}\NormalTok{(:foreigncall, :(:jl_alloc_array_}\FloatTok{1}\NormalTok{d), }\DataTypeTok{Array}\NormalTok{\{}\DataTypeTok{Float64}\NormalTok{,}\FloatTok{1}\NormalTok{\}, svec(}\DataTypeTok{Any}\NormalTok{, }\DataTypeTok{Int64}\NormalTok{), }\DataTypeTok{Array}\NormalTok{\{}\DataTypeTok{Float64}\NormalTok{,}\FloatTok{1}\NormalTok{\}, }\FloatTok{0}\NormalTok{, }\FloatTok{10}\NormalTok{, }\FloatTok{0}\NormalTok{))}
        \CommentTok{# meta: pop location}
        \CommentTok{# meta: pop location}
        \CommentTok{# meta: pop location}
        \CommentTok{# meta: pop location}
        \NormalTok{$(}\DataTypeTok{Expr}\NormalTok{(:inbounds, :pop))}
        \NormalTok{num_records = $(}\DataTypeTok{Expr}\NormalTok{(:invoke, MethodInstance }\KeywordTok{for} \NormalTok{rand!(::MersenneTwister, ::}\DataTypeTok{Array}\NormalTok{\{}\DataTypeTok{Float64}\NormalTok{,}\FloatTok{1}\NormalTok{\}, ::}\DataTypeTok{Int64}\NormalTok{, ::}\DataTypeTok{Type}\NormalTok{\{Base.Random.CloseOpen\}), :(Base.Random.rand!), :(Base.Random.GLOBAL_RNG), SSAValue(}\FloatTok{1}\NormalTok{), :((Base.arraylen)(SSAValue(}\FloatTok{1}\NormalTok{))::}\DataTypeTok{Int64}\NormalTok{), :(Base.Random.CloseOpen))) }\CommentTok{# line 33:}
        \NormalTok{(Person <: Main.AbstractStudent)::}\DataTypeTok{Bool}
        \NormalTok{goto }\FloatTok{22} \CommentTok{# line 34:}
        \FloatTok{22}\NormalTok{:  }\CommentTok{# line 36:}
        \KeywordTok{return} \NormalTok{num_records}
    \KeywordTok{end}\NormalTok{))=>}\DataTypeTok{Array}\NormalTok{\{}\DataTypeTok{Float64}\NormalTok{,}\FloatTok{1}\NormalTok{\}}
\end{Highlighting}
\end{Shaded}

we do not get a multiplication by 2.

    But the key thing to see from the typed code is that the ``branches''
(the \texttt{if} statements) all compiled away. Since types are known at
compile time (remember, functions specialize on types), the dispatch of
\texttt{likes\_music} is known at compile-time. But this means, since
the result is directly inferred from the dispatch, the boolean value
\texttt{true/false} is known at compile time. This means that the
compiler can directly infer the answer to all of these checks, and will
use this information to skip them at runtime.

This is the distinction between compile-time information and runtime
information. At compile-time, what is known is:

\begin{enumerate}
\def\labelenumi{\arabic{enumi})}
\item
  The types of the inputs
\item
  Any types which can be inferred from the input types (via
  type-stability)
\item
  The function dispatches that will be internally called (from types
  which have been inferred)
\end{enumerate}

Note that what cannot be inferred by the compiler is the information in
fields. Information in fields is strictly runtime information. This is
easy to see since there is no way for the compiler to know that person's
name was ``Miguel'': that's ephemeral and part of the type instance we
just created.

Thus by putting our information into our functions and dispatches, we
are actually giving the compiler more information to perform more
optimizations. Therefore using this ``action-based design'', we are
actually giving the compiler leeway to perform many extra optimizations
on our code as long as we define our interfaces by the actions that are
used. Of course, at the ``very bottom'' our algorithms have to use the
fields of the types, but the full interface can then be built up using a
simple set of functions which in many cases with replace runtime data
with constants.

    \subsection{Traits and THTT}\label{traits-and-thtt}

    What we just saw is a ``trait''. Traits are compile-time designations
about types which are distinct from their abstract hierarchy.
\texttt{likes\_music} is a trait which designates which people like
music, and it could in cases not be defined using the abstract types.
For example, we can, using dispatch, create a \texttt{WeirdStudent}
which does not like music, and that will still be compile-time
information which is fully optimized. This means that these small
functions which have constant return values allow for compile-time
inheritance of behavior, and these traits don't have to be tied to
abstract types (all of our examples were on \texttt{AbstracPerson}, but
we could've said a \texttt{GPUArray} likes music if we felt like it!).
Traits are multiple-inheritance for type systems.

Traits can be more refined than just \texttt{true/false}. This can be
done by having the return be a type itself. For example, we can create
music genre types:

    \begin{Verbatim}[commandchars=\\\{\}]
{\color{incolor}In [{\color{incolor}23}]:} \PY{k}{abstract} \PY{k}{type} \PY{n}{MusicGenres} \PY{k}{end}
         \PY{k}{abstract} \PY{k}{type} \PY{n}{RockGenre} \PY{o}{\PYZlt{}:} \PY{n}{MusicGenres} \PY{k}{end}
         \PY{n}{struct} \PY{n}{ClassicRock} \PY{o}{\PYZlt{}:} \PY{n}{RockGenre} \PY{k}{end}
         \PY{n}{struct} \PY{n}{AltRock} \PY{o}{\PYZlt{}:} \PY{n}{RockGenre} \PY{k}{end}
         \PY{n}{struct} \PY{n}{Classical} \PY{o}{\PYZlt{}:} \PY{n}{MusicGenres} \PY{k}{end}
\end{Verbatim}

    These ``simple types'' are known as singleton types. This means that we
can have traits like:

    \begin{Verbatim}[commandchars=\\\{\}]
{\color{incolor}In [{\color{incolor}24}]:} \PY{n}{favorite\PYZus{}genre}\PY{p}{(}\PY{n}{x}\PY{o}{::}\PY{n}{AbstractPerson}\PY{p}{)} \PY{o}{=} \PY{n}{ClassicRock}\PY{p}{(}\PY{p}{)}
         \PY{n}{favorite\PYZus{}genre}\PY{p}{(}\PY{n}{x}\PY{o}{::}\PY{n}{MusicStudent}\PY{p}{)} \PY{o}{=} \PY{n}{Classical}\PY{p}{(}\PY{p}{)}
         \PY{n}{favorite\PYZus{}genre}\PY{p}{(}\PY{n}{x}\PY{o}{::}\PY{n}{AbstractTeacher}\PY{p}{)} \PY{o}{=} \PY{n}{AltRock}\PY{p}{(}\PY{p}{)}
\end{Verbatim}

            \begin{Verbatim}[commandchars=\\\{\}]
{\color{outcolor}Out[{\color{outcolor}24}]:} favorite\_genre (generic function with 3 methods)
\end{Verbatim}
        
    This means that we can do checks like
\texttt{if typeof(favorite\_genre(x)) \textless{}: RockGenre}, and this
information will all compile away.

    This gives us all of the tools we need to compile the most efficient
code, and structure our code around types/actions/dispatch to get high
performance out. The last thing we need is syntactic sugar. Since traits
are compile-time information, the compiler could in theory dispatch on
them. While this is currently not part of Julia, it's scheduled to be
part of a future version of Julia (2.0?). The design for this (since
Julia is written in Julia!) is known as the Tim Holy Trait Trick (THTT),
named after its inventor. It's described in detail
\href{https://github.com/mauro3/Traits.jl\#dispatch-on-traits}{on this
page}. But in the end, macros can make this easier. A package which
implements trait-dispatch is
\href{https://github.com/mauro3/SimpleTraits.jl}{SimpleTraits.jl}, which
allows you to dispatch on a trait \texttt{IsNice} like:

    \begin{Verbatim}[commandchars=\\\{\}]
{\color{incolor}In [{\color{incolor} }]:} \PY{n+nd}{@traitfn} \PY{n}{ft}\PY{p}{(}\PY{n}{x}\PY{o}{::}\PY{o}{::}\PY{n}{IsNice}\PY{p}{)} \PY{o}{=} \PY{l+s}{\PYZdq{}}\PY{l+s}{V}\PY{l+s}{e}\PY{l+s}{r}\PY{l+s}{y}\PY{l+s}{ }\PY{l+s}{n}\PY{l+s}{i}\PY{l+s}{c}\PY{l+s}{e}\PY{l+s}{!}\PY{l+s}{\PYZdq{}}
        \PY{n+nd}{@traitfn} \PY{n}{ft}\PY{p}{(}\PY{n}{x}\PY{o}{::}\PY{o}{::}\PY{p}{(}\PY{o}{!}\PY{n}{IsNice}\PY{p}{)}\PY{p}{)} \PY{o}{=} \PY{l+s}{\PYZdq{}}\PY{l+s}{N}\PY{l+s}{o}\PY{l+s}{t}\PY{l+s}{ }\PY{l+s}{s}\PY{l+s}{o}\PY{l+s}{ }\PY{l+s}{n}\PY{l+s}{i}\PY{l+s}{c}\PY{l+s}{e}\PY{l+s}{!}\PY{l+s}{\PYZdq{}}
\end{Verbatim}

    \subsection{Composition vs
Inheritance}\label{composition-vs-inheritance}

    The last remark that is needed is a discussion of composition vs
inheritance. While the previous discussions have all explained why
``information not in fields'' makes structural relations compile-time
information and increases the efficiency. However, there are cases where
we want to share runtime structure. Thus the great debate of composition
vs inheritance, comes up.

Composition vs inheritance isn't a Julia issue, it's a long debate in
object-oriented programming. The idea is that, inheritance is inherently
(pun-inteded) inflexible. It forces an ``is a'' relation: A inherits
from B means A is a B, and adds a few things. It copies behavior from
something defined elsewere. This is a recipe for havoc. Here's a few
links which discuss this in more detail:

https://softwareengineering.stackexchange.com/questions/134097/why-should-i-prefer-composition-over-inheritance
https://en.wikipedia.org/wiki/Composition\_over\_inheritance
https://www.thoughtworks.com/insights/blog/composition-vs-inheritance-how-choose

So if possible, give composition a try. Say you have \texttt{MyType},
and it has some function \texttt{f} defined on it. This means that you
can extend \texttt{MyType} by making it a field in another type:

    \begin{Verbatim}[commandchars=\\\{\}]
{\color{incolor}In [{\color{incolor} }]:} \PY{n}{mutable} \PY{n}{struct} \PY{n}{MyType2}
            \PY{n}{mt}\PY{o}{::}\PY{n}{MyType}
            \PY{o}{.}\PY{o}{.}\PY{o}{.} \PY{c}{\PYZsh{} Other stuff}
        \PY{k}{end} 
        
        \PY{n}{f}\PY{p}{(}\PY{n}{mt2}\PY{o}{::}\PY{n}{MyType2}\PY{p}{)} \PY{o}{=} \PY{n}{f}\PY{p}{(}\PY{n}{mt2}\PY{o}{.}\PY{n}{mt}\PY{p}{)}
\end{Verbatim}

    The pro here is that it's explicit: you've made the choice for each
extension. The con is that this can require some extra code, though this
can be automated by metaprogramming.

    What if you really really really want inheritance of fields? There are
solutions via metaprogramming. One simple solution is the \texttt{@def}
macro.

    \begin{Verbatim}[commandchars=\\\{\}]
{\color{incolor}In [{\color{incolor}25}]:} \PY{k}{macro} \PY{n}{def}\PY{p}{(}\PY{n}{name}\PY{p}{,} \PY{n}{definition}\PY{p}{)}
               \PY{k}{return} \PY{k}{quote}
                   \PY{k}{macro} \PY{o}{\PYZdl{}}\PY{n}{name}\PY{p}{(}\PY{p}{)}
                       \PY{n}{esc}\PY{p}{(}\PY{o}{\PYZdl{}}\PY{p}{(}\PY{k+kt}{Expr}\PY{p}{(}\PY{o}{:}\PY{k}{quote}\PY{p}{,} \PY{n}{definition}\PY{p}{)}\PY{p}{)}\PY{p}{)}
                   \PY{k}{end}
               \PY{k}{end}
           \PY{k}{end}
\end{Verbatim}

            \begin{Verbatim}[commandchars=\\\{\}]
{\color{outcolor}Out[{\color{outcolor}25}]:} @def (macro with 1 method)
\end{Verbatim}
        
    This macro is very simple. What it does is compile-time copy/paste. For
example:

    \begin{Verbatim}[commandchars=\\\{\}]
{\color{incolor}In [{\color{incolor}26}]:} \PY{n+nd}{@def} \PY{n}{give\PYZus{}it\PYZus{}a\PYZus{}name} \PY{k}{begin}
           \PY{n}{a} \PY{o}{=} \PY{l+m+mi}{2}
           \PY{n}{println}\PY{p}{(}\PY{n}{a}\PY{p}{)}
         \PY{k}{end}
\end{Verbatim}

            \begin{Verbatim}[commandchars=\\\{\}]
{\color{outcolor}Out[{\color{outcolor}26}]:} @give\_it\_a\_name (macro with 1 method)
\end{Verbatim}
        
    defines a macro \texttt{@give\_it\_a\_name} that will paste in those two
lines of code wherever it is used. For example, the reused fields of
Optim.jl's solvers could be put into an \texttt{@def}:

    \begin{Verbatim}[commandchars=\\\{\}]
{\color{incolor}In [{\color{incolor}27}]:} \PY{n+nd}{@def} \PY{n}{add\PYZus{}generic\PYZus{}fields} \PY{k}{begin}
                 \PY{n}{method\PYZus{}string}\PY{o}{::}\PY{n}{String}
                 \PY{n}{n}\PY{o}{::}\PY{k+kt}{Int64}
                 \PY{n}{x}\PY{o}{::}\PY{k+kt}{Array}\PY{p}{\PYZob{}}\PY{n}{T}\PY{p}{\PYZcb{}}
                 \PY{n}{f\PYZus{}x}\PY{o}{::}\PY{n}{T}
                 \PY{n}{f\PYZus{}calls}\PY{o}{::}\PY{k+kt}{Int64}
                 \PY{n}{g\PYZus{}calls}\PY{o}{::}\PY{k+kt}{Int64}
                 \PY{n}{h\PYZus{}calls}\PY{o}{::}\PY{k+kt}{Int64}
         \PY{k}{end}
\end{Verbatim}

            \begin{Verbatim}[commandchars=\\\{\}]
{\color{outcolor}Out[{\color{outcolor}27}]:} @add\_generic\_fields (macro with 1 method)
\end{Verbatim}
        
    and those fields can be copied around with

    \begin{Verbatim}[commandchars=\\\{\}]
{\color{incolor}In [{\color{incolor}28}]:} \PY{n}{mutable} \PY{n}{struct} \PY{n}{LBFGSState}\PY{p}{\PYZob{}}\PY{n}{T}\PY{p}{\PYZcb{}}
             \PY{n+nd}{@add\PYZus{}generic\PYZus{}fields}
             \PY{n}{x\PYZus{}previous}\PY{o}{::}\PY{k+kt}{Array}\PY{p}{\PYZob{}}\PY{n}{T}\PY{p}{\PYZcb{}}
             \PY{n}{g}\PY{o}{::}\PY{k+kt}{Array}\PY{p}{\PYZob{}}\PY{n}{T}\PY{p}{\PYZcb{}}
             \PY{n}{g\PYZus{}previous}\PY{o}{::}\PY{k+kt}{Array}\PY{p}{\PYZob{}}\PY{n}{T}\PY{p}{\PYZcb{}}
             \PY{n}{rho}\PY{o}{::}\PY{k+kt}{Array}\PY{p}{\PYZob{}}\PY{n}{T}\PY{p}{\PYZcb{}}
             \PY{c}{\PYZsh{} ... more fields ... }
         \PY{k}{end}
\end{Verbatim}

    Because \texttt{@def} works at compile-time, there is no cost associated
with this. Similar metaprogramming can be used to build an ``inheritance
feature'' for Julia. One package which does this is
\href{https://github.com/tbreloff/ConcreteAbstractions.jl}{ConcreteAbstractions.jl}
which allows you to add fields to abstract types and make the child
types inherit the fields:

    \begin{Verbatim}[commandchars=\\\{\}]
{\color{incolor}In [{\color{incolor} }]:} \PY{c}{\PYZsh{} The abstract type}
        \PY{n+nd}{@base} \PY{n}{mutable} \PY{n}{struct} \PY{n}{AbstractFoo}\PY{p}{\PYZob{}}\PY{n}{T}\PY{p}{\PYZcb{}}
            \PY{n}{a}
            \PY{n}{b}\PY{o}{::}\PY{k+kt}{Int}
            \PY{n}{c}\PY{o}{::}\PY{n}{T}
            \PY{n}{d}\PY{o}{::}\PY{k+kt}{Vector}\PY{p}{\PYZob{}}\PY{n}{T}\PY{p}{\PYZcb{}}
        \PY{k}{end}
        
        \PY{c}{\PYZsh{} Inheritance}
        \PY{n+nd}{@extend} \PY{n}{mutable} \PY{n}{struct} \PY{n}{Foo} \PY{o}{\PYZlt{}:} \PY{n}{AbstractFoo}
            \PY{n+nb}{e}\PY{o}{::}\PY{n}{T}
        \PY{k}{end}
\end{Verbatim}

    where the \texttt{@extend} macro generates the type-definition:

    \begin{Verbatim}[commandchars=\\\{\}]
{\color{incolor}In [{\color{incolor} }]:} \PY{n}{mutable} \PY{n}{struct} \PY{n}{Foo}\PY{p}{\PYZob{}}\PY{n}{T}\PY{p}{\PYZcb{}} \PY{o}{\PYZlt{}:} \PY{n}{AbstractFoo}
            \PY{n}{a}
            \PY{n}{b}\PY{o}{::}\PY{k+kt}{Int}
            \PY{n}{c}\PY{o}{::}\PY{n}{T}
            \PY{n}{d}\PY{o}{::}\PY{k+kt}{Vector}\PY{p}{\PYZob{}}\PY{n}{T}\PY{p}{\PYZcb{}}
            \PY{n+nb}{e}\PY{o}{::}\PY{n}{T}
        \PY{k}{end}
\end{Verbatim}

    But it's just a package? Well, that's the beauty of Julia. Most of Julia
is written in Julia, and Julia code is first class and performant (here,
this is all at compile-time, so again runtime is not affected at all).
Honestly, if something ever gets added to Julia's Base library for this,
it will likely look very similar, and the only real difference to the
user will be that the compiler will directly recognize the keywords,
meaning you would use base and extend instead of \texttt{@base} and
\texttt{@extend}. So if you have something that really really really
needs inheritance, go for it: there's no downsides to using a package +
macro for this. But you should really try other means to reduce the
runtime information and build a more performant and more Julian
architecture first.

    \subsection{Conclusion}\label{conclusion}

Programming for type systems has a different architecture than
object-oriented systems. Instead of being oriented around the objects
and their fields, type-dispatch systems are oriented around the actions
of types. Using traits, multiple inheritance behavior can be given.
Using this structure, the compiler can have maximal information, and use
this to optimize the code. But also, this directly generalizes the vast
majority of the code to not be ``implementation-dependent'', allowing
for duck-typed code to be fully performance, with all of the details
handled by dispatch/traits/abstract types. The end result is flexible,
generic, and high performance code.


    % Add a bibliography block to the postdoc
    
    
    
    \end{document}
