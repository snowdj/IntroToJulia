
% Default to the notebook output style

    


% Inherit from the specified cell style.




    
\documentclass[11pt]{article}

    
    
    \usepackage[T1]{fontenc}
    % Nicer default font than Computer Modern for most use cases
    \usepackage{palatino}

    % Basic figure setup, for now with no caption control since it's done
    % automatically by Pandoc (which extracts ![](path) syntax from Markdown).
    \usepackage{graphicx}
    % We will generate all images so they have a width \maxwidth. This means
    % that they will get their normal width if they fit onto the page, but
    % are scaled down if they would overflow the margins.
    \makeatletter
    \def\maxwidth{\ifdim\Gin@nat@width>\linewidth\linewidth
    \else\Gin@nat@width\fi}
    \makeatother
    \let\Oldincludegraphics\includegraphics
    % Set max figure width to be 80% of text width, for now hardcoded.
    \renewcommand{\includegraphics}[1]{\Oldincludegraphics[width=.8\maxwidth]{#1}}
    % Ensure that by default, figures have no caption (until we provide a
    % proper Figure object with a Caption API and a way to capture that
    % in the conversion process - todo).
    \usepackage{caption}
    \DeclareCaptionLabelFormat{nolabel}{}
    \captionsetup{labelformat=nolabel}

    \usepackage{adjustbox} % Used to constrain images to a maximum size 
    \usepackage{xcolor} % Allow colors to be defined
    \usepackage{enumerate} % Needed for markdown enumerations to work
    \usepackage{geometry} % Used to adjust the document margins
    \usepackage{amsmath} % Equations
    \usepackage{amssymb} % Equations
    \usepackage{textcomp} % defines textquotesingle
    % Hack from http://tex.stackexchange.com/a/47451/13684:
    \AtBeginDocument{%
        \def\PYZsq{\textquotesingle}% Upright quotes in Pygmentized code
    }
    \usepackage{upquote} % Upright quotes for verbatim code
    \usepackage{eurosym} % defines \euro
    \usepackage[mathletters]{ucs} % Extended unicode (utf-8) support
    \usepackage[utf8x]{inputenc} % Allow utf-8 characters in the tex document
    \usepackage{fancyvrb} % verbatim replacement that allows latex
    \usepackage{grffile} % extends the file name processing of package graphics 
                         % to support a larger range 
    % The hyperref package gives us a pdf with properly built
    % internal navigation ('pdf bookmarks' for the table of contents,
    % internal cross-reference links, web links for URLs, etc.)
    \usepackage{hyperref}
    \usepackage{longtable} % longtable support required by pandoc >1.10
    \usepackage{booktabs}  % table support for pandoc > 1.12.2
    \usepackage[normalem]{ulem} % ulem is needed to support strikethroughs (\sout)
                                % normalem makes italics be italics, not underlines
    

    
    
    % Colors for the hyperref package
    \definecolor{urlcolor}{rgb}{0,.145,.698}
    \definecolor{linkcolor}{rgb}{.71,0.21,0.01}
    \definecolor{citecolor}{rgb}{.12,.54,.11}

    % ANSI colors
    \definecolor{ansi-black}{HTML}{3E424D}
    \definecolor{ansi-black-intense}{HTML}{282C36}
    \definecolor{ansi-red}{HTML}{E75C58}
    \definecolor{ansi-red-intense}{HTML}{B22B31}
    \definecolor{ansi-green}{HTML}{00A250}
    \definecolor{ansi-green-intense}{HTML}{007427}
    \definecolor{ansi-yellow}{HTML}{DDB62B}
    \definecolor{ansi-yellow-intense}{HTML}{B27D12}
    \definecolor{ansi-blue}{HTML}{208FFB}
    \definecolor{ansi-blue-intense}{HTML}{0065CA}
    \definecolor{ansi-magenta}{HTML}{D160C4}
    \definecolor{ansi-magenta-intense}{HTML}{A03196}
    \definecolor{ansi-cyan}{HTML}{60C6C8}
    \definecolor{ansi-cyan-intense}{HTML}{258F8F}
    \definecolor{ansi-white}{HTML}{C5C1B4}
    \definecolor{ansi-white-intense}{HTML}{A1A6B2}

    % commands and environments needed by pandoc snippets
    % extracted from the output of `pandoc -s`
    \providecommand{\tightlist}{%
      \setlength{\itemsep}{0pt}\setlength{\parskip}{0pt}}
    \DefineVerbatimEnvironment{Highlighting}{Verbatim}{commandchars=\\\{\}}
    % Add ',fontsize=\small' for more characters per line
    \newenvironment{Shaded}{}{}
    \newcommand{\KeywordTok}[1]{\textcolor[rgb]{0.00,0.44,0.13}{\textbf{{#1}}}}
    \newcommand{\DataTypeTok}[1]{\textcolor[rgb]{0.56,0.13,0.00}{{#1}}}
    \newcommand{\DecValTok}[1]{\textcolor[rgb]{0.25,0.63,0.44}{{#1}}}
    \newcommand{\BaseNTok}[1]{\textcolor[rgb]{0.25,0.63,0.44}{{#1}}}
    \newcommand{\FloatTok}[1]{\textcolor[rgb]{0.25,0.63,0.44}{{#1}}}
    \newcommand{\CharTok}[1]{\textcolor[rgb]{0.25,0.44,0.63}{{#1}}}
    \newcommand{\StringTok}[1]{\textcolor[rgb]{0.25,0.44,0.63}{{#1}}}
    \newcommand{\CommentTok}[1]{\textcolor[rgb]{0.38,0.63,0.69}{\textit{{#1}}}}
    \newcommand{\OtherTok}[1]{\textcolor[rgb]{0.00,0.44,0.13}{{#1}}}
    \newcommand{\AlertTok}[1]{\textcolor[rgb]{1.00,0.00,0.00}{\textbf{{#1}}}}
    \newcommand{\FunctionTok}[1]{\textcolor[rgb]{0.02,0.16,0.49}{{#1}}}
    \newcommand{\RegionMarkerTok}[1]{{#1}}
    \newcommand{\ErrorTok}[1]{\textcolor[rgb]{1.00,0.00,0.00}{\textbf{{#1}}}}
    \newcommand{\NormalTok}[1]{{#1}}
    
    % Additional commands for more recent versions of Pandoc
    \newcommand{\ConstantTok}[1]{\textcolor[rgb]{0.53,0.00,0.00}{{#1}}}
    \newcommand{\SpecialCharTok}[1]{\textcolor[rgb]{0.25,0.44,0.63}{{#1}}}
    \newcommand{\VerbatimStringTok}[1]{\textcolor[rgb]{0.25,0.44,0.63}{{#1}}}
    \newcommand{\SpecialStringTok}[1]{\textcolor[rgb]{0.73,0.40,0.53}{{#1}}}
    \newcommand{\ImportTok}[1]{{#1}}
    \newcommand{\DocumentationTok}[1]{\textcolor[rgb]{0.73,0.13,0.13}{\textit{{#1}}}}
    \newcommand{\AnnotationTok}[1]{\textcolor[rgb]{0.38,0.63,0.69}{\textbf{\textit{{#1}}}}}
    \newcommand{\CommentVarTok}[1]{\textcolor[rgb]{0.38,0.63,0.69}{\textbf{\textit{{#1}}}}}
    \newcommand{\VariableTok}[1]{\textcolor[rgb]{0.10,0.09,0.49}{{#1}}}
    \newcommand{\ControlFlowTok}[1]{\textcolor[rgb]{0.00,0.44,0.13}{\textbf{{#1}}}}
    \newcommand{\OperatorTok}[1]{\textcolor[rgb]{0.40,0.40,0.40}{{#1}}}
    \newcommand{\BuiltInTok}[1]{{#1}}
    \newcommand{\ExtensionTok}[1]{{#1}}
    \newcommand{\PreprocessorTok}[1]{\textcolor[rgb]{0.74,0.48,0.00}{{#1}}}
    \newcommand{\AttributeTok}[1]{\textcolor[rgb]{0.49,0.56,0.16}{{#1}}}
    \newcommand{\InformationTok}[1]{\textcolor[rgb]{0.38,0.63,0.69}{\textbf{\textit{{#1}}}}}
    \newcommand{\WarningTok}[1]{\textcolor[rgb]{0.38,0.63,0.69}{\textbf{\textit{{#1}}}}}
    
    
    % Define a nice break command that doesn't care if a line doesn't already
    % exist.
    \def\br{\hspace*{\fill} \\* }
    % Math Jax compatability definitions
    \def\gt{>}
    \def\lt{<}
    % Document parameters
    \title{PlotsJL}
    
    
    

    % Pygments definitions
    
\makeatletter
\def\PY@reset{\let\PY@it=\relax \let\PY@bf=\relax%
    \let\PY@ul=\relax \let\PY@tc=\relax%
    \let\PY@bc=\relax \let\PY@ff=\relax}
\def\PY@tok#1{\csname PY@tok@#1\endcsname}
\def\PY@toks#1+{\ifx\relax#1\empty\else%
    \PY@tok{#1}\expandafter\PY@toks\fi}
\def\PY@do#1{\PY@bc{\PY@tc{\PY@ul{%
    \PY@it{\PY@bf{\PY@ff{#1}}}}}}}
\def\PY#1#2{\PY@reset\PY@toks#1+\relax+\PY@do{#2}}

\expandafter\def\csname PY@tok@gd\endcsname{\def\PY@tc##1{\textcolor[rgb]{0.63,0.00,0.00}{##1}}}
\expandafter\def\csname PY@tok@gu\endcsname{\let\PY@bf=\textbf\def\PY@tc##1{\textcolor[rgb]{0.50,0.00,0.50}{##1}}}
\expandafter\def\csname PY@tok@gt\endcsname{\def\PY@tc##1{\textcolor[rgb]{0.00,0.27,0.87}{##1}}}
\expandafter\def\csname PY@tok@gs\endcsname{\let\PY@bf=\textbf}
\expandafter\def\csname PY@tok@gr\endcsname{\def\PY@tc##1{\textcolor[rgb]{1.00,0.00,0.00}{##1}}}
\expandafter\def\csname PY@tok@cm\endcsname{\let\PY@it=\textit\def\PY@tc##1{\textcolor[rgb]{0.25,0.50,0.50}{##1}}}
\expandafter\def\csname PY@tok@vg\endcsname{\def\PY@tc##1{\textcolor[rgb]{0.10,0.09,0.49}{##1}}}
\expandafter\def\csname PY@tok@vi\endcsname{\def\PY@tc##1{\textcolor[rgb]{0.10,0.09,0.49}{##1}}}
\expandafter\def\csname PY@tok@vm\endcsname{\def\PY@tc##1{\textcolor[rgb]{0.10,0.09,0.49}{##1}}}
\expandafter\def\csname PY@tok@mh\endcsname{\def\PY@tc##1{\textcolor[rgb]{0.40,0.40,0.40}{##1}}}
\expandafter\def\csname PY@tok@cs\endcsname{\let\PY@it=\textit\def\PY@tc##1{\textcolor[rgb]{0.25,0.50,0.50}{##1}}}
\expandafter\def\csname PY@tok@ge\endcsname{\let\PY@it=\textit}
\expandafter\def\csname PY@tok@vc\endcsname{\def\PY@tc##1{\textcolor[rgb]{0.10,0.09,0.49}{##1}}}
\expandafter\def\csname PY@tok@il\endcsname{\def\PY@tc##1{\textcolor[rgb]{0.40,0.40,0.40}{##1}}}
\expandafter\def\csname PY@tok@go\endcsname{\def\PY@tc##1{\textcolor[rgb]{0.53,0.53,0.53}{##1}}}
\expandafter\def\csname PY@tok@cp\endcsname{\def\PY@tc##1{\textcolor[rgb]{0.74,0.48,0.00}{##1}}}
\expandafter\def\csname PY@tok@gi\endcsname{\def\PY@tc##1{\textcolor[rgb]{0.00,0.63,0.00}{##1}}}
\expandafter\def\csname PY@tok@gh\endcsname{\let\PY@bf=\textbf\def\PY@tc##1{\textcolor[rgb]{0.00,0.00,0.50}{##1}}}
\expandafter\def\csname PY@tok@ni\endcsname{\let\PY@bf=\textbf\def\PY@tc##1{\textcolor[rgb]{0.60,0.60,0.60}{##1}}}
\expandafter\def\csname PY@tok@nl\endcsname{\def\PY@tc##1{\textcolor[rgb]{0.63,0.63,0.00}{##1}}}
\expandafter\def\csname PY@tok@nn\endcsname{\let\PY@bf=\textbf\def\PY@tc##1{\textcolor[rgb]{0.00,0.00,1.00}{##1}}}
\expandafter\def\csname PY@tok@no\endcsname{\def\PY@tc##1{\textcolor[rgb]{0.53,0.00,0.00}{##1}}}
\expandafter\def\csname PY@tok@na\endcsname{\def\PY@tc##1{\textcolor[rgb]{0.49,0.56,0.16}{##1}}}
\expandafter\def\csname PY@tok@nb\endcsname{\def\PY@tc##1{\textcolor[rgb]{0.00,0.50,0.00}{##1}}}
\expandafter\def\csname PY@tok@nc\endcsname{\let\PY@bf=\textbf\def\PY@tc##1{\textcolor[rgb]{0.00,0.00,1.00}{##1}}}
\expandafter\def\csname PY@tok@nd\endcsname{\def\PY@tc##1{\textcolor[rgb]{0.67,0.13,1.00}{##1}}}
\expandafter\def\csname PY@tok@ne\endcsname{\let\PY@bf=\textbf\def\PY@tc##1{\textcolor[rgb]{0.82,0.25,0.23}{##1}}}
\expandafter\def\csname PY@tok@nf\endcsname{\def\PY@tc##1{\textcolor[rgb]{0.00,0.00,1.00}{##1}}}
\expandafter\def\csname PY@tok@si\endcsname{\let\PY@bf=\textbf\def\PY@tc##1{\textcolor[rgb]{0.73,0.40,0.53}{##1}}}
\expandafter\def\csname PY@tok@s2\endcsname{\def\PY@tc##1{\textcolor[rgb]{0.73,0.13,0.13}{##1}}}
\expandafter\def\csname PY@tok@nt\endcsname{\let\PY@bf=\textbf\def\PY@tc##1{\textcolor[rgb]{0.00,0.50,0.00}{##1}}}
\expandafter\def\csname PY@tok@nv\endcsname{\def\PY@tc##1{\textcolor[rgb]{0.10,0.09,0.49}{##1}}}
\expandafter\def\csname PY@tok@s1\endcsname{\def\PY@tc##1{\textcolor[rgb]{0.73,0.13,0.13}{##1}}}
\expandafter\def\csname PY@tok@dl\endcsname{\def\PY@tc##1{\textcolor[rgb]{0.73,0.13,0.13}{##1}}}
\expandafter\def\csname PY@tok@ch\endcsname{\let\PY@it=\textit\def\PY@tc##1{\textcolor[rgb]{0.25,0.50,0.50}{##1}}}
\expandafter\def\csname PY@tok@m\endcsname{\def\PY@tc##1{\textcolor[rgb]{0.40,0.40,0.40}{##1}}}
\expandafter\def\csname PY@tok@gp\endcsname{\let\PY@bf=\textbf\def\PY@tc##1{\textcolor[rgb]{0.00,0.00,0.50}{##1}}}
\expandafter\def\csname PY@tok@sh\endcsname{\def\PY@tc##1{\textcolor[rgb]{0.73,0.13,0.13}{##1}}}
\expandafter\def\csname PY@tok@ow\endcsname{\let\PY@bf=\textbf\def\PY@tc##1{\textcolor[rgb]{0.67,0.13,1.00}{##1}}}
\expandafter\def\csname PY@tok@sx\endcsname{\def\PY@tc##1{\textcolor[rgb]{0.00,0.50,0.00}{##1}}}
\expandafter\def\csname PY@tok@bp\endcsname{\def\PY@tc##1{\textcolor[rgb]{0.00,0.50,0.00}{##1}}}
\expandafter\def\csname PY@tok@c1\endcsname{\let\PY@it=\textit\def\PY@tc##1{\textcolor[rgb]{0.25,0.50,0.50}{##1}}}
\expandafter\def\csname PY@tok@fm\endcsname{\def\PY@tc##1{\textcolor[rgb]{0.00,0.00,1.00}{##1}}}
\expandafter\def\csname PY@tok@o\endcsname{\def\PY@tc##1{\textcolor[rgb]{0.40,0.40,0.40}{##1}}}
\expandafter\def\csname PY@tok@kc\endcsname{\let\PY@bf=\textbf\def\PY@tc##1{\textcolor[rgb]{0.00,0.50,0.00}{##1}}}
\expandafter\def\csname PY@tok@c\endcsname{\let\PY@it=\textit\def\PY@tc##1{\textcolor[rgb]{0.25,0.50,0.50}{##1}}}
\expandafter\def\csname PY@tok@mf\endcsname{\def\PY@tc##1{\textcolor[rgb]{0.40,0.40,0.40}{##1}}}
\expandafter\def\csname PY@tok@err\endcsname{\def\PY@bc##1{\setlength{\fboxsep}{0pt}\fcolorbox[rgb]{1.00,0.00,0.00}{1,1,1}{\strut ##1}}}
\expandafter\def\csname PY@tok@mb\endcsname{\def\PY@tc##1{\textcolor[rgb]{0.40,0.40,0.40}{##1}}}
\expandafter\def\csname PY@tok@ss\endcsname{\def\PY@tc##1{\textcolor[rgb]{0.10,0.09,0.49}{##1}}}
\expandafter\def\csname PY@tok@sr\endcsname{\def\PY@tc##1{\textcolor[rgb]{0.73,0.40,0.53}{##1}}}
\expandafter\def\csname PY@tok@mo\endcsname{\def\PY@tc##1{\textcolor[rgb]{0.40,0.40,0.40}{##1}}}
\expandafter\def\csname PY@tok@kd\endcsname{\let\PY@bf=\textbf\def\PY@tc##1{\textcolor[rgb]{0.00,0.50,0.00}{##1}}}
\expandafter\def\csname PY@tok@mi\endcsname{\def\PY@tc##1{\textcolor[rgb]{0.40,0.40,0.40}{##1}}}
\expandafter\def\csname PY@tok@kn\endcsname{\let\PY@bf=\textbf\def\PY@tc##1{\textcolor[rgb]{0.00,0.50,0.00}{##1}}}
\expandafter\def\csname PY@tok@cpf\endcsname{\let\PY@it=\textit\def\PY@tc##1{\textcolor[rgb]{0.25,0.50,0.50}{##1}}}
\expandafter\def\csname PY@tok@kr\endcsname{\let\PY@bf=\textbf\def\PY@tc##1{\textcolor[rgb]{0.00,0.50,0.00}{##1}}}
\expandafter\def\csname PY@tok@s\endcsname{\def\PY@tc##1{\textcolor[rgb]{0.73,0.13,0.13}{##1}}}
\expandafter\def\csname PY@tok@kp\endcsname{\def\PY@tc##1{\textcolor[rgb]{0.00,0.50,0.00}{##1}}}
\expandafter\def\csname PY@tok@w\endcsname{\def\PY@tc##1{\textcolor[rgb]{0.73,0.73,0.73}{##1}}}
\expandafter\def\csname PY@tok@kt\endcsname{\def\PY@tc##1{\textcolor[rgb]{0.69,0.00,0.25}{##1}}}
\expandafter\def\csname PY@tok@sc\endcsname{\def\PY@tc##1{\textcolor[rgb]{0.73,0.13,0.13}{##1}}}
\expandafter\def\csname PY@tok@sb\endcsname{\def\PY@tc##1{\textcolor[rgb]{0.73,0.13,0.13}{##1}}}
\expandafter\def\csname PY@tok@sa\endcsname{\def\PY@tc##1{\textcolor[rgb]{0.73,0.13,0.13}{##1}}}
\expandafter\def\csname PY@tok@k\endcsname{\let\PY@bf=\textbf\def\PY@tc##1{\textcolor[rgb]{0.00,0.50,0.00}{##1}}}
\expandafter\def\csname PY@tok@se\endcsname{\let\PY@bf=\textbf\def\PY@tc##1{\textcolor[rgb]{0.73,0.40,0.13}{##1}}}
\expandafter\def\csname PY@tok@sd\endcsname{\let\PY@it=\textit\def\PY@tc##1{\textcolor[rgb]{0.73,0.13,0.13}{##1}}}

\def\PYZbs{\char`\\}
\def\PYZus{\char`\_}
\def\PYZob{\char`\{}
\def\PYZcb{\char`\}}
\def\PYZca{\char`\^}
\def\PYZam{\char`\&}
\def\PYZlt{\char`\<}
\def\PYZgt{\char`\>}
\def\PYZsh{\char`\#}
\def\PYZpc{\char`\%}
\def\PYZdl{\char`\$}
\def\PYZhy{\char`\-}
\def\PYZsq{\char`\'}
\def\PYZdq{\char`\"}
\def\PYZti{\char`\~}
% for compatibility with earlier versions
\def\PYZat{@}
\def\PYZlb{[}
\def\PYZrb{]}
\makeatother


    % Exact colors from NB
    \definecolor{incolor}{rgb}{0.0, 0.0, 0.5}
    \definecolor{outcolor}{rgb}{0.545, 0.0, 0.0}



    
    % Prevent overflowing lines due to hard-to-break entities
    \sloppy 
    % Setup hyperref package
    \hypersetup{
      breaklinks=true,  % so long urls are correctly broken across lines
      colorlinks=true,
      urlcolor=urlcolor,
      linkcolor=linkcolor,
      citecolor=citecolor,
      }
    % Slightly bigger margins than the latex defaults
    
    \geometry{verbose,tmargin=1in,bmargin=1in,lmargin=1in,rmargin=1in}
    
    

    \begin{document}
    
    
    \maketitle
    
    

    
    \subsection{An Introduction to
Plots.jl}\label{an-introduction-to-plots.jl}

    \subsection{Idea}\label{idea}

Plots.jl is a non-traditional plotting library

\begin{itemize}
\itemsep1pt\parskip0pt\parsep0pt
\item
  It does not implement a ``plotting backend'' itself, it's a plotting
  API
\item
  The API is easily extendable via recipes
\end{itemize}

    \subsection{Backends}\label{backends}

Plots.jl uses other plotting libraries as backends

\begin{itemize}
\itemsep1pt\parskip0pt\parsep0pt
\item
  PyPlot (matplotlib): Slow but dependable
\item
  GR: Feature-rich and fast, but new
\item
  Plotly/PlotlyJS: Interactive and good for web
\item
  PGFPlots: Native LaTeX rendering
\item
  UnicodePlots: Plots to unicode for no-display situations
\end{itemize}

    \subsection{Using Backends}\label{using-backends}

To switch backends, you simply use the name of the library:
https://juliaplots.github.io/backends/

    \begin{Verbatim}[commandchars=\\\{\}]
{\color{incolor}In [{\color{incolor}2}]:} \PY{k}{using} \PY{n}{Plots}
        \PY{n}{pyplot}\PY{p}{(}\PY{p}{)} \PY{c}{\PYZsh{} Turns on the PyPlot backend}
        \PY{n}{plot}\PY{p}{(}\PY{n}{rand}\PY{p}{(}\PY{l+m+mi}{4}\PY{p}{,}\PY{l+m+mi}{4}\PY{p}{)}\PY{p}{)}
\end{Verbatim}

    \begin{Verbatim}[commandchars=\\\{\}]
{\color{incolor}In [{\color{incolor}4}]:} \PY{n}{gr}\PY{p}{(}\PY{p}{)}
        \PY{n}{plot}\PY{p}{(}\PY{n}{rand}\PY{p}{(}\PY{l+m+mi}{4}\PY{p}{,}\PY{l+m+mi}{4}\PY{p}{)}\PY{p}{)}
\end{Verbatim}

    \begin{Verbatim}[commandchars=\\\{\}]
{\color{incolor}In [{\color{incolor}7}]:} \PY{n}{plotly}\PY{p}{(}\PY{p}{)}
        \PY{n}{plot}\PY{p}{(}\PY{n}{rand}\PY{p}{(}\PY{l+m+mi}{4}\PY{p}{,}\PY{l+m+mi}{4}\PY{p}{)}\PY{p}{)}
\end{Verbatim}

    \subsection{Attributes}\label{attributes}

The attributes work with each of the backends:
https://juliaplots.github.io/attributes/

Compatibility of attributes is found in this chart:
https://juliaplots.github.io/supported/

I find it easiest to use this page to find the right attributes:
https://juliaplots.github.io/examples/pyplot/

    \begin{Verbatim}[commandchars=\\\{\}]
{\color{incolor}In [{\color{incolor}15}]:} \PY{n}{pyplot}\PY{p}{(}\PY{p}{)}
         \PY{n}{plot}\PY{p}{(}\PY{n}{rand}\PY{p}{(}\PY{l+m+mi}{4}\PY{p}{,}\PY{l+m+mi}{4}\PY{p}{)}\PY{p}{,}\PY{n}{title}\PY{o}{=}\PY{l+s}{\PYZdq{}}\PY{l+s}{T}\PY{l+s}{e}\PY{l+s}{s}\PY{l+s}{t}\PY{l+s}{ }\PY{l+s}{T}\PY{l+s}{i}\PY{l+s}{t}\PY{l+s}{l}\PY{l+s}{e}\PY{l+s}{\PYZdq{}}\PY{p}{,}\PY{n}{label}\PY{o}{=}\PY{p}{[}\PY{l+s}{\PYZdq{}}\PY{l+s}{F}\PY{l+s}{i}\PY{l+s}{r}\PY{l+s}{s}\PY{l+s}{t}\PY{l+s}{\PYZdq{}} \PY{l+s}{\PYZdq{}}\PY{l+s}{S}\PY{l+s}{e}\PY{l+s}{c}\PY{l+s}{o}\PY{l+s}{n}\PY{l+s}{d}\PY{l+s}{\PYZdq{}} \PY{l+s}{\PYZdq{}}\PY{l+s}{T}\PY{l+s}{h}\PY{l+s}{i}\PY{l+s}{r}\PY{l+s}{d}\PY{l+s}{\PYZdq{}} \PY{l+s}{\PYZdq{}}\PY{l+s}{F}\PY{l+s}{o}\PY{l+s}{u}\PY{l+s}{r}\PY{l+s}{t}\PY{l+s}{h}\PY{l+s}{\PYZdq{}}\PY{p}{]}\PY{p}{)}
\end{Verbatim}

    \begin{Verbatim}[commandchars=\\\{\}]
{\color{incolor}In [{\color{incolor}16}]:} \PY{n}{gr}\PY{p}{(}\PY{p}{)}
         \PY{n}{plot}\PY{p}{(}\PY{n}{rand}\PY{p}{(}\PY{l+m+mi}{4}\PY{p}{,}\PY{l+m+mi}{4}\PY{p}{)}\PY{p}{,}\PY{n}{title}\PY{o}{=}\PY{l+s}{\PYZdq{}}\PY{l+s}{T}\PY{l+s}{e}\PY{l+s}{s}\PY{l+s}{t}\PY{l+s}{ }\PY{l+s}{T}\PY{l+s}{i}\PY{l+s}{t}\PY{l+s}{l}\PY{l+s}{e}\PY{l+s}{\PYZdq{}}\PY{p}{,}\PY{n}{label}\PY{o}{=}\PY{p}{[}\PY{l+s}{\PYZdq{}}\PY{l+s}{F}\PY{l+s}{i}\PY{l+s}{r}\PY{l+s}{s}\PY{l+s}{t}\PY{l+s}{\PYZdq{}} \PY{l+s}{\PYZdq{}}\PY{l+s}{S}\PY{l+s}{e}\PY{l+s}{c}\PY{l+s}{o}\PY{l+s}{n}\PY{l+s}{d}\PY{l+s}{\PYZdq{}} \PY{l+s}{\PYZdq{}}\PY{l+s}{T}\PY{l+s}{h}\PY{l+s}{i}\PY{l+s}{r}\PY{l+s}{d}\PY{l+s}{\PYZdq{}} \PY{l+s}{\PYZdq{}}\PY{l+s}{F}\PY{l+s}{o}\PY{l+s}{u}\PY{l+s}{r}\PY{l+s}{t}\PY{l+s}{h}\PY{l+s}{\PYZdq{}}\PY{p}{]}\PY{p}{)}
\end{Verbatim}

    \begin{Verbatim}[commandchars=\\\{\}]
{\color{incolor}In [{\color{incolor}17}]:} \PY{n}{plotly}\PY{p}{(}\PY{p}{)}
         \PY{n}{plot}\PY{p}{(}\PY{n}{rand}\PY{p}{(}\PY{l+m+mi}{4}\PY{p}{,}\PY{l+m+mi}{4}\PY{p}{)}\PY{p}{,}\PY{n}{title}\PY{o}{=}\PY{l+s}{\PYZdq{}}\PY{l+s}{T}\PY{l+s}{e}\PY{l+s}{s}\PY{l+s}{t}\PY{l+s}{ }\PY{l+s}{T}\PY{l+s}{i}\PY{l+s}{t}\PY{l+s}{l}\PY{l+s}{e}\PY{l+s}{\PYZdq{}}\PY{p}{,}\PY{n}{label}\PY{o}{=}\PY{p}{[}\PY{l+s}{\PYZdq{}}\PY{l+s}{F}\PY{l+s}{i}\PY{l+s}{r}\PY{l+s}{s}\PY{l+s}{t}\PY{l+s}{\PYZdq{}} \PY{l+s}{\PYZdq{}}\PY{l+s}{S}\PY{l+s}{e}\PY{l+s}{c}\PY{l+s}{o}\PY{l+s}{n}\PY{l+s}{d}\PY{l+s}{\PYZdq{}} \PY{l+s}{\PYZdq{}}\PY{l+s}{T}\PY{l+s}{h}\PY{l+s}{i}\PY{l+s}{r}\PY{l+s}{d}\PY{l+s}{\PYZdq{}} \PY{l+s}{\PYZdq{}}\PY{l+s}{F}\PY{l+s}{o}\PY{l+s}{u}\PY{l+s}{r}\PY{l+s}{t}\PY{l+s}{h}\PY{l+s}{\PYZdq{}}\PY{p}{]}\PY{p}{)}
\end{Verbatim}

    \subsection{Animations}\label{animations}

Any plot can be animated

    \begin{Verbatim}[commandchars=\\\{\}]
{\color{incolor}In [{\color{incolor}15}]:} \PY{c}{\PYZsh{} initialize the attractor}
         \PY{n}{n} \PY{o}{=} \PY{l+m+mi}{1500}
         \PY{n}{dt} \PY{o}{=} \PY{l+m+mf}{0.02}
         \PY{n}{σ}\PY{p}{,} \PY{n}{ρ}\PY{p}{,} \PY{n}{β} \PY{o}{=} \PY{l+m+mf}{10.}\PY{p}{,} \PY{l+m+mf}{28.}\PY{p}{,} \PY{l+m+mi}{8}\PY{o}{/}\PY{l+m+mi}{3}
         \PY{n}{x}\PY{p}{,} \PY{n}{y}\PY{p}{,} \PY{n}{z} \PY{o}{=} \PY{l+m+mf}{1.}\PY{p}{,} \PY{l+m+mf}{1.}\PY{p}{,} \PY{l+m+mf}{1.}
         
         \PY{c}{\PYZsh{} initialize a 3D plot with 1 empty series}
         \PY{n}{plt} \PY{o}{=} \PY{n}{path3d}\PY{p}{(}\PY{l+m+mi}{1}\PY{p}{,} \PY{n}{xlim}\PY{o}{=}\PY{p}{(}\PY{o}{\PYZhy{}}\PY{l+m+mi}{25}\PY{p}{,}\PY{l+m+mi}{25}\PY{p}{)}\PY{p}{,} \PY{n}{ylim}\PY{o}{=}\PY{p}{(}\PY{o}{\PYZhy{}}\PY{l+m+mi}{25}\PY{p}{,}\PY{l+m+mi}{25}\PY{p}{)}\PY{p}{,} \PY{n}{zlim}\PY{o}{=}\PY{p}{(}\PY{l+m+mi}{0}\PY{p}{,}\PY{l+m+mi}{50}\PY{p}{)}\PY{p}{,}
                         \PY{n}{xlab} \PY{o}{=} \PY{l+s}{\PYZdq{}}\PY{l+s}{x}\PY{l+s}{\PYZdq{}}\PY{p}{,} \PY{n}{ylab} \PY{o}{=} \PY{l+s}{\PYZdq{}}\PY{l+s}{y}\PY{l+s}{\PYZdq{}}\PY{p}{,} \PY{n}{zlab} \PY{o}{=} \PY{l+s}{\PYZdq{}}\PY{l+s}{z}\PY{l+s}{\PYZdq{}}\PY{p}{,}
                         \PY{n}{title} \PY{o}{=} \PY{l+s}{\PYZdq{}}\PY{l+s}{L}\PY{l+s}{o}\PY{l+s}{r}\PY{l+s}{e}\PY{l+s}{n}\PY{l+s}{z}\PY{l+s}{ }\PY{l+s}{A}\PY{l+s}{t}\PY{l+s}{t}\PY{l+s}{r}\PY{l+s}{a}\PY{l+s}{c}\PY{l+s}{t}\PY{l+s}{o}\PY{l+s}{r}\PY{l+s}{\PYZdq{}}\PY{p}{,} \PY{n}{marker} \PY{o}{=} \PY{l+m+mi}{1}\PY{p}{)}
         
         \PY{c}{\PYZsh{} build an animated gif, saving every 10th frame}
         \PY{n+nd}{@gif} \PY{k}{for} \PY{n}{i}\PY{o}{=}\PY{l+m+mi}{1}\PY{o}{:}\PY{n}{n}
             \PY{n}{dx} \PY{o}{=} \PY{n}{σ}\PY{o}{*}\PY{p}{(}\PY{n}{y} \PY{o}{\PYZhy{}} \PY{n}{x}\PY{p}{)}     \PY{p}{;} \PY{n}{x} \PY{o}{+=} \PY{n}{dt} \PY{o}{*} \PY{n}{dx}
             \PY{n}{dy} \PY{o}{=} \PY{n}{x}\PY{o}{*}\PY{p}{(}\PY{n}{ρ} \PY{o}{\PYZhy{}} \PY{n}{z}\PY{p}{)} \PY{o}{\PYZhy{}} \PY{n}{y} \PY{p}{;} \PY{n}{y} \PY{o}{+=} \PY{n}{dt} \PY{o}{*} \PY{n}{dy}
             \PY{n}{dz} \PY{o}{=} \PY{n}{x}\PY{o}{*}\PY{n}{y} \PY{o}{\PYZhy{}} \PY{n}{β}\PY{o}{*}\PY{n}{z}     \PY{p}{;} \PY{n}{z} \PY{o}{+=} \PY{n}{dt} \PY{o}{*} \PY{n}{dz}
             \PY{n}{push!}\PY{p}{(}\PY{n}{plt}\PY{p}{,} \PY{n}{x}\PY{p}{,} \PY{n}{y}\PY{p}{,} \PY{n}{z}\PY{p}{)}
         \PY{k}{end} \PY{n}{every} \PY{l+m+mi}{10}
\end{Verbatim}

    \begin{Verbatim}[commandchars=\\\{\}]
INFO: Saved animation to C:\textbackslash{}Users\textbackslash{}Chris\textbackslash{}.julia\textbackslash{}v0.5\textbackslash{}IntroToJulia\textbackslash{}notebooks\textbackslash{}tmp.gif

    \end{Verbatim}

            \begin{Verbatim}[commandchars=\\\{\}]
{\color{outcolor}Out[{\color{outcolor}15}]:} Plots.AnimatedGif("C:\textbackslash{}\textbackslash{}Users\textbackslash{}\textbackslash{}Chris\textbackslash{}\textbackslash{}.julia\textbackslash{}\textbackslash{}v0.5\textbackslash{}\textbackslash{}IntroToJulia\textbackslash{}\textbackslash{}notebooks\textbackslash{}\textbackslash{}tmp.gif")
\end{Verbatim}
        
    \subsection{Recipes}\label{recipes}

Recipes are abstract instructions for how to ``build a plot'' from data.
There are multiple kinds of recipes. In execution order:

\begin{itemize}
\itemsep1pt\parskip0pt\parsep0pt
\item
  User Recipes: Provides dispatches to plotting
\item
  Type Recipes: Says how to interpret the data of an abstract type
\item
  Plot Recipes: A pre-processing recipe which builds a set of series
  plots and defaults
\item
  Series Recipes: What most would think of as a ``type of plot'',
  i.e.~scatter, histogram, etc.
\end{itemize}

Since these extend Plots.jl itself, all of Plots.jl is accessible from
the plotting commands that these make, and these recipes are accessible
from each other.

{[}Series recipes are used to extend the compatibility of backends
itself!{]}

\href{https://juliaplots.github.io/ecosystem/}{Check out of the Plots
Ecosystem!}

    \subsection{Type Recipe Example}\label{type-recipe-example}

    \begin{Verbatim}[commandchars=\\\{\}]
{\color{incolor}In [{\color{incolor}24}]:} \PY{k}{using} \PY{n}{DifferentialEquations}
         \PY{n}{sol} \PY{o}{=} \PY{n}{solve}\PY{p}{(}\PY{n}{prob\PYZus{}ode\PYZus{}linear}\PY{p}{)}
         \PY{n+nd}{@show} \PY{n}{typeof}\PY{p}{(}\PY{n}{sol}\PY{p}{)}
         \PY{n}{plot}\PY{p}{(}\PY{n}{sol}\PY{p}{,}\PY{n}{title}\PY{o}{=}\PY{l+s}{\PYZdq{}}\PY{l+s}{T}\PY{l+s}{h}\PY{l+s}{e}\PY{l+s}{ }\PY{l+s}{A}\PY{l+s}{t}\PY{l+s}{t}\PY{l+s}{r}\PY{l+s}{i}\PY{l+s}{b}\PY{l+s}{u}\PY{l+s}{t}\PY{l+s}{e}\PY{l+s}{s}\PY{l+s}{ }\PY{l+s}{S}\PY{l+s}{t}\PY{l+s}{i}\PY{l+s}{l}\PY{l+s}{l}\PY{l+s}{ }\PY{l+s}{W}\PY{l+s}{o}\PY{l+s}{r}\PY{l+s}{k}\PY{l+s}{\PYZdq{}}\PY{p}{)}
\end{Verbatim}

    \begin{Verbatim}[commandchars=\\\{\}]
typeof(sol) = DifferentialEquations.ODESolution

    \end{Verbatim}

    \subsection{Plot and Type Recipes
Together}\label{plot-and-type-recipes-together}

StatsPlots provides a type recipe for how to read DataFrames, and a
series recipe \texttt{marginalhist} which puts together histograms into
a cohesive larger plot

    \begin{Verbatim}[commandchars=\\\{\}]
{\color{incolor}In [{\color{incolor}2}]:} \PY{k}{using} \PY{n}{RDatasets}\PY{p}{,} \PY{n}{StatPlots}\PY{p}{,} \PY{n}{Plots}
        \PY{n}{iris} \PY{o}{=} \PY{n}{dataset}\PY{p}{(}\PY{l+s}{\PYZdq{}}\PY{l+s}{d}\PY{l+s}{a}\PY{l+s}{t}\PY{l+s}{a}\PY{l+s}{s}\PY{l+s}{e}\PY{l+s}{t}\PY{l+s}{s}\PY{l+s}{\PYZdq{}}\PY{p}{,}\PY{l+s}{\PYZdq{}}\PY{l+s}{i}\PY{l+s}{r}\PY{l+s}{i}\PY{l+s}{s}\PY{l+s}{\PYZdq{}}\PY{p}{)}
        \PY{n}{marginalhist}\PY{p}{(}\PY{n}{iris}\PY{p}{,} \PY{o}{:}\PY{n}{PetalLength}\PY{p}{,} \PY{o}{:}\PY{n}{PetalWidth}\PY{p}{)}
\end{Verbatim}

    \begin{Verbatim}[commandchars=\\\{\}]
WARNING: using Plots.PyPlot in module Main conflicts with an existing identifier.

    \end{Verbatim}

    \begin{Verbatim}[commandchars=\\\{\}]
{\color{incolor}In [{\color{incolor}7}]:} \PY{n}{M} \PY{o}{=} \PY{n}{randn}\PY{p}{(}\PY{l+m+mi}{1000}\PY{p}{,}\PY{l+m+mi}{4}\PY{p}{)}
        \PY{n}{M}\PY{p}{[}\PY{o}{:}\PY{p}{,}\PY{l+m+mi}{2}\PY{p}{]} \PY{o}{+=} \PY{l+m+mf}{0.8}\PY{n}{sqrt}\PY{p}{(}\PY{n}{abs}\PY{p}{(}\PY{n}{M}\PY{p}{[}\PY{o}{:}\PY{p}{,}\PY{l+m+mi}{1}\PY{p}{]}\PY{p}{)}\PY{p}{)} \PY{o}{\PYZhy{}} \PY{l+m+mf}{0.5}\PY{n}{M}\PY{p}{[}\PY{o}{:}\PY{p}{,}\PY{l+m+mi}{3}\PY{p}{]} \PY{o}{+} \PY{l+m+mi}{5}
        \PY{n}{M}\PY{p}{[}\PY{o}{:}\PY{p}{,}\PY{l+m+mi}{3}\PY{p}{]} \PY{o}{\PYZhy{}=} \PY{l+m+mf}{0.7}\PY{n}{M}\PY{p}{[}\PY{o}{:}\PY{p}{,}\PY{l+m+mi}{1}\PY{p}{]}\PY{o}{.\PYZca{}}\PY{l+m+mi}{2} \PY{o}{+} \PY{l+m+mi}{2}
        \PY{n}{corrplot}\PY{p}{(}\PY{n}{M}\PY{p}{,} \PY{n}{label} \PY{o}{=} \PY{p}{[}\PY{l+s}{\PYZdq{}}\PY{l+s}{x}\PY{l+s+si}{\PYZdl{}i}\PY{l+s}{\PYZdq{}} \PY{k}{for} \PY{n}{i}\PY{o}{=}\PY{l+m+mi}{1}\PY{o}{:}\PY{l+m+mi}{4}\PY{p}{]}\PY{p}{)}
\end{Verbatim}

    \begin{Verbatim}[commandchars=\\\{\}]
{\color{incolor}In [{\color{incolor}14}]:} \PY{k}{import} \PY{n}{RDatasets}
         \PY{n}{pyplot}\PY{p}{(}\PY{p}{)}
         \PY{n}{singers} \PY{o}{=} \PY{n}{RDatasets}\PY{o}{.}\PY{n}{dataset}\PY{p}{(}\PY{l+s}{\PYZdq{}}\PY{l+s}{l}\PY{l+s}{a}\PY{l+s}{t}\PY{l+s}{t}\PY{l+s}{i}\PY{l+s}{c}\PY{l+s}{e}\PY{l+s}{\PYZdq{}}\PY{p}{,}\PY{l+s}{\PYZdq{}}\PY{l+s}{s}\PY{l+s}{i}\PY{l+s}{n}\PY{l+s}{g}\PY{l+s}{e}\PY{l+s}{r}\PY{l+s}{\PYZdq{}}\PY{p}{)}
         \PY{n}{violin}\PY{p}{(}\PY{n}{singers}\PY{p}{,}\PY{o}{:}\PY{n}{VoicePart}\PY{p}{,}\PY{o}{:}\PY{n}{Height}\PY{p}{,}\PY{n}{marker}\PY{o}{=}\PY{p}{(}\PY{l+m+mf}{0.2}\PY{p}{,}\PY{o}{:}\PY{n}{blue}\PY{p}{,}\PY{n}{stroke}\PY{p}{(}\PY{l+m+mi}{0}\PY{p}{)}\PY{p}{)}\PY{p}{)}
         \PY{n}{boxplot!}\PY{p}{(}\PY{n}{singers}\PY{p}{,}\PY{o}{:}\PY{n}{VoicePart}\PY{p}{,}\PY{o}{:}\PY{n}{Height}\PY{p}{,}\PY{n}{marker}\PY{o}{=}\PY{p}{(}\PY{l+m+mf}{0.3}\PY{p}{,}\PY{o}{:}\PY{n}{orange}\PY{p}{,}\PY{n}{stroke}\PY{p}{(}\PY{l+m+mi}{2}\PY{p}{)}\PY{p}{)}\PY{p}{)}
\end{Verbatim}

    \subsection{Series Type}\label{series-type}

A series type allows you to define an entirely new way of visualizing
data into backends.

    \begin{Verbatim}[commandchars=\\\{\}]
{\color{incolor}In [{\color{incolor}9}]:} \PY{n}{groupedbar}\PY{p}{(}\PY{n}{rand}\PY{p}{(}\PY{l+m+mi}{10}\PY{p}{,}\PY{l+m+mi}{3}\PY{p}{)}\PY{p}{,} \PY{n}{bar\PYZus{}position} \PY{o}{=} \PY{o}{:}\PY{n}{dodge}\PY{p}{,} \PY{n}{bar\PYZus{}width}\PY{o}{=}\PY{l+m+mf}{0.7}\PY{p}{)}
\end{Verbatim}

    \begin{Verbatim}[commandchars=\\\{\}]
{\color{incolor}In [{\color{incolor}11}]:} \PY{n}{gr}\PY{p}{(}\PY{p}{)}
         \PY{n}{groupedbar}\PY{p}{(}\PY{n}{rand}\PY{p}{(}\PY{l+m+mi}{10}\PY{p}{,}\PY{l+m+mi}{3}\PY{p}{)}\PY{p}{,} \PY{n}{bar\PYZus{}position} \PY{o}{=} \PY{o}{:}\PY{n}{dodge}\PY{p}{,} \PY{n}{bar\PYZus{}width}\PY{o}{=}\PY{l+m+mf}{0.7}\PY{p}{)}
\end{Verbatim}

    \subsection{Project: Regression Plot}\label{project-regression-plot}

Make a beautiful plot of your regression:

\begin{itemize}
\itemsep1pt\parskip0pt\parsep0pt
\item
  Plot the values as a scatter plot
\item
  Use the mutating plot (\texttt{plot!}) to add the linear regression
  line over the scatter plot
\item
  Use Loess.jl to build a smoothed line, and see how that plots vs your
  linear regression
\item
  Add a title, label the two lines in a legend, and label the \texttt{x}
  and \texttt{y} axis
\item
  Try some other backends: which one do you like the best?
\end{itemize}


    % Add a bibliography block to the postdoc
    
    
    
    \end{document}
