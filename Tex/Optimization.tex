
% Default to the notebook output style

    


% Inherit from the specified cell style.




    
\documentclass[11pt]{article}

    
    
    \usepackage[T1]{fontenc}
    % Nicer default font than Computer Modern for most use cases
    \usepackage{palatino}

    % Basic figure setup, for now with no caption control since it's done
    % automatically by Pandoc (which extracts ![](path) syntax from Markdown).
    \usepackage{graphicx}
    % We will generate all images so they have a width \maxwidth. This means
    % that they will get their normal width if they fit onto the page, but
    % are scaled down if they would overflow the margins.
    \makeatletter
    \def\maxwidth{\ifdim\Gin@nat@width>\linewidth\linewidth
    \else\Gin@nat@width\fi}
    \makeatother
    \let\Oldincludegraphics\includegraphics
    % Set max figure width to be 80% of text width, for now hardcoded.
    \renewcommand{\includegraphics}[1]{\Oldincludegraphics[width=.8\maxwidth]{#1}}
    % Ensure that by default, figures have no caption (until we provide a
    % proper Figure object with a Caption API and a way to capture that
    % in the conversion process - todo).
    \usepackage{caption}
    \DeclareCaptionLabelFormat{nolabel}{}
    \captionsetup{labelformat=nolabel}

    \usepackage{adjustbox} % Used to constrain images to a maximum size 
    \usepackage{xcolor} % Allow colors to be defined
    \usepackage{enumerate} % Needed for markdown enumerations to work
    \usepackage{geometry} % Used to adjust the document margins
    \usepackage{amsmath} % Equations
    \usepackage{amssymb} % Equations
    \usepackage{textcomp} % defines textquotesingle
    % Hack from http://tex.stackexchange.com/a/47451/13684:
    \AtBeginDocument{%
        \def\PYZsq{\textquotesingle}% Upright quotes in Pygmentized code
    }
    \usepackage{upquote} % Upright quotes for verbatim code
    \usepackage{eurosym} % defines \euro
    \usepackage[mathletters]{ucs} % Extended unicode (utf-8) support
    \usepackage[utf8x]{inputenc} % Allow utf-8 characters in the tex document
    \usepackage{fancyvrb} % verbatim replacement that allows latex
    \usepackage{grffile} % extends the file name processing of package graphics 
                         % to support a larger range 
    % The hyperref package gives us a pdf with properly built
    % internal navigation ('pdf bookmarks' for the table of contents,
    % internal cross-reference links, web links for URLs, etc.)
    \usepackage{hyperref}
    \usepackage{longtable} % longtable support required by pandoc >1.10
    \usepackage{booktabs}  % table support for pandoc > 1.12.2
    \usepackage[normalem]{ulem} % ulem is needed to support strikethroughs (\sout)
                                % normalem makes italics be italics, not underlines
    

    
    
    % Colors for the hyperref package
    \definecolor{urlcolor}{rgb}{0,.145,.698}
    \definecolor{linkcolor}{rgb}{.71,0.21,0.01}
    \definecolor{citecolor}{rgb}{.12,.54,.11}

    % ANSI colors
    \definecolor{ansi-black}{HTML}{3E424D}
    \definecolor{ansi-black-intense}{HTML}{282C36}
    \definecolor{ansi-red}{HTML}{E75C58}
    \definecolor{ansi-red-intense}{HTML}{B22B31}
    \definecolor{ansi-green}{HTML}{00A250}
    \definecolor{ansi-green-intense}{HTML}{007427}
    \definecolor{ansi-yellow}{HTML}{DDB62B}
    \definecolor{ansi-yellow-intense}{HTML}{B27D12}
    \definecolor{ansi-blue}{HTML}{208FFB}
    \definecolor{ansi-blue-intense}{HTML}{0065CA}
    \definecolor{ansi-magenta}{HTML}{D160C4}
    \definecolor{ansi-magenta-intense}{HTML}{A03196}
    \definecolor{ansi-cyan}{HTML}{60C6C8}
    \definecolor{ansi-cyan-intense}{HTML}{258F8F}
    \definecolor{ansi-white}{HTML}{C5C1B4}
    \definecolor{ansi-white-intense}{HTML}{A1A6B2}

    % commands and environments needed by pandoc snippets
    % extracted from the output of `pandoc -s`
    \providecommand{\tightlist}{%
      \setlength{\itemsep}{0pt}\setlength{\parskip}{0pt}}
    \DefineVerbatimEnvironment{Highlighting}{Verbatim}{commandchars=\\\{\}}
    % Add ',fontsize=\small' for more characters per line
    \newenvironment{Shaded}{}{}
    \newcommand{\KeywordTok}[1]{\textcolor[rgb]{0.00,0.44,0.13}{\textbf{{#1}}}}
    \newcommand{\DataTypeTok}[1]{\textcolor[rgb]{0.56,0.13,0.00}{{#1}}}
    \newcommand{\DecValTok}[1]{\textcolor[rgb]{0.25,0.63,0.44}{{#1}}}
    \newcommand{\BaseNTok}[1]{\textcolor[rgb]{0.25,0.63,0.44}{{#1}}}
    \newcommand{\FloatTok}[1]{\textcolor[rgb]{0.25,0.63,0.44}{{#1}}}
    \newcommand{\CharTok}[1]{\textcolor[rgb]{0.25,0.44,0.63}{{#1}}}
    \newcommand{\StringTok}[1]{\textcolor[rgb]{0.25,0.44,0.63}{{#1}}}
    \newcommand{\CommentTok}[1]{\textcolor[rgb]{0.38,0.63,0.69}{\textit{{#1}}}}
    \newcommand{\OtherTok}[1]{\textcolor[rgb]{0.00,0.44,0.13}{{#1}}}
    \newcommand{\AlertTok}[1]{\textcolor[rgb]{1.00,0.00,0.00}{\textbf{{#1}}}}
    \newcommand{\FunctionTok}[1]{\textcolor[rgb]{0.02,0.16,0.49}{{#1}}}
    \newcommand{\RegionMarkerTok}[1]{{#1}}
    \newcommand{\ErrorTok}[1]{\textcolor[rgb]{1.00,0.00,0.00}{\textbf{{#1}}}}
    \newcommand{\NormalTok}[1]{{#1}}
    
    % Additional commands for more recent versions of Pandoc
    \newcommand{\ConstantTok}[1]{\textcolor[rgb]{0.53,0.00,0.00}{{#1}}}
    \newcommand{\SpecialCharTok}[1]{\textcolor[rgb]{0.25,0.44,0.63}{{#1}}}
    \newcommand{\VerbatimStringTok}[1]{\textcolor[rgb]{0.25,0.44,0.63}{{#1}}}
    \newcommand{\SpecialStringTok}[1]{\textcolor[rgb]{0.73,0.40,0.53}{{#1}}}
    \newcommand{\ImportTok}[1]{{#1}}
    \newcommand{\DocumentationTok}[1]{\textcolor[rgb]{0.73,0.13,0.13}{\textit{{#1}}}}
    \newcommand{\AnnotationTok}[1]{\textcolor[rgb]{0.38,0.63,0.69}{\textbf{\textit{{#1}}}}}
    \newcommand{\CommentVarTok}[1]{\textcolor[rgb]{0.38,0.63,0.69}{\textbf{\textit{{#1}}}}}
    \newcommand{\VariableTok}[1]{\textcolor[rgb]{0.10,0.09,0.49}{{#1}}}
    \newcommand{\ControlFlowTok}[1]{\textcolor[rgb]{0.00,0.44,0.13}{\textbf{{#1}}}}
    \newcommand{\OperatorTok}[1]{\textcolor[rgb]{0.40,0.40,0.40}{{#1}}}
    \newcommand{\BuiltInTok}[1]{{#1}}
    \newcommand{\ExtensionTok}[1]{{#1}}
    \newcommand{\PreprocessorTok}[1]{\textcolor[rgb]{0.74,0.48,0.00}{{#1}}}
    \newcommand{\AttributeTok}[1]{\textcolor[rgb]{0.49,0.56,0.16}{{#1}}}
    \newcommand{\InformationTok}[1]{\textcolor[rgb]{0.38,0.63,0.69}{\textbf{\textit{{#1}}}}}
    \newcommand{\WarningTok}[1]{\textcolor[rgb]{0.38,0.63,0.69}{\textbf{\textit{{#1}}}}}
    
    
    % Define a nice break command that doesn't care if a line doesn't already
    % exist.
    \def\br{\hspace*{\fill} \\* }
    % Math Jax compatability definitions
    \def\gt{>}
    \def\lt{<}
    % Document parameters
    \title{Optimization}
    
    
    

    % Pygments definitions
    
\makeatletter
\def\PY@reset{\let\PY@it=\relax \let\PY@bf=\relax%
    \let\PY@ul=\relax \let\PY@tc=\relax%
    \let\PY@bc=\relax \let\PY@ff=\relax}
\def\PY@tok#1{\csname PY@tok@#1\endcsname}
\def\PY@toks#1+{\ifx\relax#1\empty\else%
    \PY@tok{#1}\expandafter\PY@toks\fi}
\def\PY@do#1{\PY@bc{\PY@tc{\PY@ul{%
    \PY@it{\PY@bf{\PY@ff{#1}}}}}}}
\def\PY#1#2{\PY@reset\PY@toks#1+\relax+\PY@do{#2}}

\expandafter\def\csname PY@tok@gd\endcsname{\def\PY@tc##1{\textcolor[rgb]{0.63,0.00,0.00}{##1}}}
\expandafter\def\csname PY@tok@gu\endcsname{\let\PY@bf=\textbf\def\PY@tc##1{\textcolor[rgb]{0.50,0.00,0.50}{##1}}}
\expandafter\def\csname PY@tok@gt\endcsname{\def\PY@tc##1{\textcolor[rgb]{0.00,0.27,0.87}{##1}}}
\expandafter\def\csname PY@tok@gs\endcsname{\let\PY@bf=\textbf}
\expandafter\def\csname PY@tok@gr\endcsname{\def\PY@tc##1{\textcolor[rgb]{1.00,0.00,0.00}{##1}}}
\expandafter\def\csname PY@tok@cm\endcsname{\let\PY@it=\textit\def\PY@tc##1{\textcolor[rgb]{0.25,0.50,0.50}{##1}}}
\expandafter\def\csname PY@tok@vg\endcsname{\def\PY@tc##1{\textcolor[rgb]{0.10,0.09,0.49}{##1}}}
\expandafter\def\csname PY@tok@vi\endcsname{\def\PY@tc##1{\textcolor[rgb]{0.10,0.09,0.49}{##1}}}
\expandafter\def\csname PY@tok@vm\endcsname{\def\PY@tc##1{\textcolor[rgb]{0.10,0.09,0.49}{##1}}}
\expandafter\def\csname PY@tok@mh\endcsname{\def\PY@tc##1{\textcolor[rgb]{0.40,0.40,0.40}{##1}}}
\expandafter\def\csname PY@tok@cs\endcsname{\let\PY@it=\textit\def\PY@tc##1{\textcolor[rgb]{0.25,0.50,0.50}{##1}}}
\expandafter\def\csname PY@tok@ge\endcsname{\let\PY@it=\textit}
\expandafter\def\csname PY@tok@vc\endcsname{\def\PY@tc##1{\textcolor[rgb]{0.10,0.09,0.49}{##1}}}
\expandafter\def\csname PY@tok@il\endcsname{\def\PY@tc##1{\textcolor[rgb]{0.40,0.40,0.40}{##1}}}
\expandafter\def\csname PY@tok@go\endcsname{\def\PY@tc##1{\textcolor[rgb]{0.53,0.53,0.53}{##1}}}
\expandafter\def\csname PY@tok@cp\endcsname{\def\PY@tc##1{\textcolor[rgb]{0.74,0.48,0.00}{##1}}}
\expandafter\def\csname PY@tok@gi\endcsname{\def\PY@tc##1{\textcolor[rgb]{0.00,0.63,0.00}{##1}}}
\expandafter\def\csname PY@tok@gh\endcsname{\let\PY@bf=\textbf\def\PY@tc##1{\textcolor[rgb]{0.00,0.00,0.50}{##1}}}
\expandafter\def\csname PY@tok@ni\endcsname{\let\PY@bf=\textbf\def\PY@tc##1{\textcolor[rgb]{0.60,0.60,0.60}{##1}}}
\expandafter\def\csname PY@tok@nl\endcsname{\def\PY@tc##1{\textcolor[rgb]{0.63,0.63,0.00}{##1}}}
\expandafter\def\csname PY@tok@nn\endcsname{\let\PY@bf=\textbf\def\PY@tc##1{\textcolor[rgb]{0.00,0.00,1.00}{##1}}}
\expandafter\def\csname PY@tok@no\endcsname{\def\PY@tc##1{\textcolor[rgb]{0.53,0.00,0.00}{##1}}}
\expandafter\def\csname PY@tok@na\endcsname{\def\PY@tc##1{\textcolor[rgb]{0.49,0.56,0.16}{##1}}}
\expandafter\def\csname PY@tok@nb\endcsname{\def\PY@tc##1{\textcolor[rgb]{0.00,0.50,0.00}{##1}}}
\expandafter\def\csname PY@tok@nc\endcsname{\let\PY@bf=\textbf\def\PY@tc##1{\textcolor[rgb]{0.00,0.00,1.00}{##1}}}
\expandafter\def\csname PY@tok@nd\endcsname{\def\PY@tc##1{\textcolor[rgb]{0.67,0.13,1.00}{##1}}}
\expandafter\def\csname PY@tok@ne\endcsname{\let\PY@bf=\textbf\def\PY@tc##1{\textcolor[rgb]{0.82,0.25,0.23}{##1}}}
\expandafter\def\csname PY@tok@nf\endcsname{\def\PY@tc##1{\textcolor[rgb]{0.00,0.00,1.00}{##1}}}
\expandafter\def\csname PY@tok@si\endcsname{\let\PY@bf=\textbf\def\PY@tc##1{\textcolor[rgb]{0.73,0.40,0.53}{##1}}}
\expandafter\def\csname PY@tok@s2\endcsname{\def\PY@tc##1{\textcolor[rgb]{0.73,0.13,0.13}{##1}}}
\expandafter\def\csname PY@tok@nt\endcsname{\let\PY@bf=\textbf\def\PY@tc##1{\textcolor[rgb]{0.00,0.50,0.00}{##1}}}
\expandafter\def\csname PY@tok@nv\endcsname{\def\PY@tc##1{\textcolor[rgb]{0.10,0.09,0.49}{##1}}}
\expandafter\def\csname PY@tok@s1\endcsname{\def\PY@tc##1{\textcolor[rgb]{0.73,0.13,0.13}{##1}}}
\expandafter\def\csname PY@tok@dl\endcsname{\def\PY@tc##1{\textcolor[rgb]{0.73,0.13,0.13}{##1}}}
\expandafter\def\csname PY@tok@ch\endcsname{\let\PY@it=\textit\def\PY@tc##1{\textcolor[rgb]{0.25,0.50,0.50}{##1}}}
\expandafter\def\csname PY@tok@m\endcsname{\def\PY@tc##1{\textcolor[rgb]{0.40,0.40,0.40}{##1}}}
\expandafter\def\csname PY@tok@gp\endcsname{\let\PY@bf=\textbf\def\PY@tc##1{\textcolor[rgb]{0.00,0.00,0.50}{##1}}}
\expandafter\def\csname PY@tok@sh\endcsname{\def\PY@tc##1{\textcolor[rgb]{0.73,0.13,0.13}{##1}}}
\expandafter\def\csname PY@tok@ow\endcsname{\let\PY@bf=\textbf\def\PY@tc##1{\textcolor[rgb]{0.67,0.13,1.00}{##1}}}
\expandafter\def\csname PY@tok@sx\endcsname{\def\PY@tc##1{\textcolor[rgb]{0.00,0.50,0.00}{##1}}}
\expandafter\def\csname PY@tok@bp\endcsname{\def\PY@tc##1{\textcolor[rgb]{0.00,0.50,0.00}{##1}}}
\expandafter\def\csname PY@tok@c1\endcsname{\let\PY@it=\textit\def\PY@tc##1{\textcolor[rgb]{0.25,0.50,0.50}{##1}}}
\expandafter\def\csname PY@tok@fm\endcsname{\def\PY@tc##1{\textcolor[rgb]{0.00,0.00,1.00}{##1}}}
\expandafter\def\csname PY@tok@o\endcsname{\def\PY@tc##1{\textcolor[rgb]{0.40,0.40,0.40}{##1}}}
\expandafter\def\csname PY@tok@kc\endcsname{\let\PY@bf=\textbf\def\PY@tc##1{\textcolor[rgb]{0.00,0.50,0.00}{##1}}}
\expandafter\def\csname PY@tok@c\endcsname{\let\PY@it=\textit\def\PY@tc##1{\textcolor[rgb]{0.25,0.50,0.50}{##1}}}
\expandafter\def\csname PY@tok@mf\endcsname{\def\PY@tc##1{\textcolor[rgb]{0.40,0.40,0.40}{##1}}}
\expandafter\def\csname PY@tok@err\endcsname{\def\PY@bc##1{\setlength{\fboxsep}{0pt}\fcolorbox[rgb]{1.00,0.00,0.00}{1,1,1}{\strut ##1}}}
\expandafter\def\csname PY@tok@mb\endcsname{\def\PY@tc##1{\textcolor[rgb]{0.40,0.40,0.40}{##1}}}
\expandafter\def\csname PY@tok@ss\endcsname{\def\PY@tc##1{\textcolor[rgb]{0.10,0.09,0.49}{##1}}}
\expandafter\def\csname PY@tok@sr\endcsname{\def\PY@tc##1{\textcolor[rgb]{0.73,0.40,0.53}{##1}}}
\expandafter\def\csname PY@tok@mo\endcsname{\def\PY@tc##1{\textcolor[rgb]{0.40,0.40,0.40}{##1}}}
\expandafter\def\csname PY@tok@kd\endcsname{\let\PY@bf=\textbf\def\PY@tc##1{\textcolor[rgb]{0.00,0.50,0.00}{##1}}}
\expandafter\def\csname PY@tok@mi\endcsname{\def\PY@tc##1{\textcolor[rgb]{0.40,0.40,0.40}{##1}}}
\expandafter\def\csname PY@tok@kn\endcsname{\let\PY@bf=\textbf\def\PY@tc##1{\textcolor[rgb]{0.00,0.50,0.00}{##1}}}
\expandafter\def\csname PY@tok@cpf\endcsname{\let\PY@it=\textit\def\PY@tc##1{\textcolor[rgb]{0.25,0.50,0.50}{##1}}}
\expandafter\def\csname PY@tok@kr\endcsname{\let\PY@bf=\textbf\def\PY@tc##1{\textcolor[rgb]{0.00,0.50,0.00}{##1}}}
\expandafter\def\csname PY@tok@s\endcsname{\def\PY@tc##1{\textcolor[rgb]{0.73,0.13,0.13}{##1}}}
\expandafter\def\csname PY@tok@kp\endcsname{\def\PY@tc##1{\textcolor[rgb]{0.00,0.50,0.00}{##1}}}
\expandafter\def\csname PY@tok@w\endcsname{\def\PY@tc##1{\textcolor[rgb]{0.73,0.73,0.73}{##1}}}
\expandafter\def\csname PY@tok@kt\endcsname{\def\PY@tc##1{\textcolor[rgb]{0.69,0.00,0.25}{##1}}}
\expandafter\def\csname PY@tok@sc\endcsname{\def\PY@tc##1{\textcolor[rgb]{0.73,0.13,0.13}{##1}}}
\expandafter\def\csname PY@tok@sb\endcsname{\def\PY@tc##1{\textcolor[rgb]{0.73,0.13,0.13}{##1}}}
\expandafter\def\csname PY@tok@sa\endcsname{\def\PY@tc##1{\textcolor[rgb]{0.73,0.13,0.13}{##1}}}
\expandafter\def\csname PY@tok@k\endcsname{\let\PY@bf=\textbf\def\PY@tc##1{\textcolor[rgb]{0.00,0.50,0.00}{##1}}}
\expandafter\def\csname PY@tok@se\endcsname{\let\PY@bf=\textbf\def\PY@tc##1{\textcolor[rgb]{0.73,0.40,0.13}{##1}}}
\expandafter\def\csname PY@tok@sd\endcsname{\let\PY@it=\textit\def\PY@tc##1{\textcolor[rgb]{0.73,0.13,0.13}{##1}}}

\def\PYZbs{\char`\\}
\def\PYZus{\char`\_}
\def\PYZob{\char`\{}
\def\PYZcb{\char`\}}
\def\PYZca{\char`\^}
\def\PYZam{\char`\&}
\def\PYZlt{\char`\<}
\def\PYZgt{\char`\>}
\def\PYZsh{\char`\#}
\def\PYZpc{\char`\%}
\def\PYZdl{\char`\$}
\def\PYZhy{\char`\-}
\def\PYZsq{\char`\'}
\def\PYZdq{\char`\"}
\def\PYZti{\char`\~}
% for compatibility with earlier versions
\def\PYZat{@}
\def\PYZlb{[}
\def\PYZrb{]}
\makeatother


    % Exact colors from NB
    \definecolor{incolor}{rgb}{0.0, 0.0, 0.5}
    \definecolor{outcolor}{rgb}{0.545, 0.0, 0.0}



    
    % Prevent overflowing lines due to hard-to-break entities
    \sloppy 
    % Setup hyperref package
    \hypersetup{
      breaklinks=true,  % so long urls are correctly broken across lines
      colorlinks=true,
      urlcolor=urlcolor,
      linkcolor=linkcolor,
      citecolor=citecolor,
      }
    % Slightly bigger margins than the latex defaults
    
    \geometry{verbose,tmargin=1in,bmargin=1in,lmargin=1in,rmargin=1in}
    
    

    \begin{document}
    
    
    \maketitle
    
    

    
    \section{Optimization}\label{optimization}

\subsection{Local Nonlinear Optimization with
Optim.jl}\label{local-nonlinear-optimization-with-optim.jl}

One of the core libraries for nonlinear optimization is Optim.jl.
Optim.jl is a lot like the standard optimizers you'd find in SciPy or
MATLAB. You give it a function and it finds the minimum. For example, if
you give it a univariate function it uses Brent's method to find the
minimum in an interval:

    \begin{Verbatim}[commandchars=\\\{\}]
{\color{incolor}In [{\color{incolor}2}]:} \PY{k}{using} \PY{n}{Optim}
        \PY{n}{f}\PY{p}{(}\PY{n}{x}\PY{p}{)} \PY{o}{=} \PY{n}{sin}\PY{p}{(}\PY{n}{x}\PY{p}{)}\PY{o}{+}\PY{n}{cos}\PY{p}{(}\PY{n}{x}\PY{p}{)}
        \PY{n}{Optim}\PY{o}{.}\PY{n}{optimize}\PY{p}{(}\PY{n}{f}\PY{p}{,}\PY{l+m+mf}{0.0}\PY{p}{,}\PY{l+m+mi}{2}\PY{n+nb}{π}\PY{p}{)} \PY{c}{\PYZsh{} Find a minimum between 0 and 2π}
\end{Verbatim}

            \begin{Verbatim}[commandchars=\\\{\}]
{\color{outcolor}Out[{\color{outcolor}2}]:} Results of Optimization Algorithm
         * Algorithm: Brent's Method
         * Search Interval: [0.000000, 6.283185]
         * Minimizer: 3.926991e+00
         * Minimum: -1.414214e+00
         * Iterations: 11
         * Convergence: max(|x - x\_upper|, |x - x\_lower|) <= 2*(1.5e-08*|x|+2.2e-16): true
         * Objective Function Calls: 12
\end{Verbatim}
        
    If you give it a function which requires vector input with scalar
output, it will give the vector local minima:

    \begin{Verbatim}[commandchars=\\\{\}]
{\color{incolor}In [{\color{incolor}4}]:} \PY{n}{f}\PY{p}{(}\PY{n}{x}\PY{p}{)} \PY{o}{=} \PY{n}{sin}\PY{p}{(}\PY{n}{x}\PY{p}{[}\PY{l+m+mi}{1}\PY{p}{]}\PY{p}{)}\PY{o}{+}\PY{n}{cos}\PY{p}{(}\PY{n}{x}\PY{p}{[}\PY{l+m+mi}{1}\PY{p}{]}\PY{o}{+}\PY{n}{x}\PY{p}{[}\PY{l+m+mi}{2}\PY{p}{]}\PY{p}{)}
        \PY{n}{Optim}\PY{o}{.}\PY{n}{optimize}\PY{p}{(}\PY{n}{f}\PY{p}{,}\PY{n}{zeros}\PY{p}{(}\PY{l+m+mi}{2}\PY{p}{)}\PY{p}{)} \PY{c}{\PYZsh{} Find a minimum starting at [0.0,0.0]}
\end{Verbatim}

            \begin{Verbatim}[commandchars=\\\{\}]
{\color{outcolor}Out[{\color{outcolor}4}]:} Results of Optimization Algorithm
         * Algorithm: Nelder-Mead
         * Starting Point: [0.0,0.0]
         * Minimizer: [-1.570684758073873,-1.5708688186478836]
         * Minimum: -2.000000e+00
         * Iterations: 49
         * Convergence: true
           *  √(Σ(yᵢ-ȳ)²)/n < 1.0e-08: true
           * Reached Maximum Number of Iterations: false
         * Objective Calls: 95
\end{Verbatim}
        
    You can refer to Optim's
\href{http://julianlsolvers.github.io/Optim.jl/latest/}{large library of
methods} and pass in a method choice to have different properties. Let's
choose BFGS:

    \begin{Verbatim}[commandchars=\\\{\}]
{\color{incolor}In [{\color{incolor}5}]:} \PY{n}{Optim}\PY{o}{.}\PY{n}{optimize}\PY{p}{(}\PY{n}{f}\PY{p}{,}\PY{n}{zeros}\PY{p}{(}\PY{l+m+mi}{2}\PY{p}{)}\PY{p}{,}\PY{n}{BFGS}\PY{p}{(}\PY{p}{)}\PY{p}{)}
\end{Verbatim}

            \begin{Verbatim}[commandchars=\\\{\}]
{\color{outcolor}Out[{\color{outcolor}5}]:} Results of Optimization Algorithm
         * Algorithm: BFGS
         * Starting Point: [0.0,0.0]
         * Minimizer: [-1.5707963314270867,-1.5707963181008544]
         * Minimum: -2.000000e+00
         * Iterations: 6
         * Convergence: true
           * |x - x'| ≤ 1.0e-32: false 
             |x - x'| = 4.10e-04 
           * |f(x) - f(x')| ≤ 1.0e-32 |f(x)|: false
             |f(x) - f(x')| = -9.67e-08 |f(x)|
           * |g(x)| ≤ 1.0e-08: true 
             |g(x)| = 4.06e-09 
           * Stopped by an increasing objective: false
           * Reached Maximum Number of Iterations: false
         * Objective Calls: 15
         * Gradient Calls: 15
\end{Verbatim}
        
    \subsection{Global Nonlinear Optimization with
BlackBoxOptim.jl}\label{global-nonlinear-optimization-with-blackboxoptim.jl}

Global optimization is provided with a native Julia implementation at
BlackBoxOptim.jl. You have to give it box constraints and tell it the
size of the input vector:

    \begin{Verbatim}[commandchars=\\\{\}]
{\color{incolor}In [{\color{incolor}7}]:} \PY{k}{using} \PY{n}{BlackBoxOptim}
        
        \PY{k}{function} \PY{n}{rosenbrock2d}\PY{p}{(}\PY{n}{x}\PY{p}{)}
          \PY{k}{return} \PY{p}{(}\PY{l+m+mf}{1.0} \PY{o}{\PYZhy{}} \PY{n}{x}\PY{p}{[}\PY{l+m+mi}{1}\PY{p}{]}\PY{p}{)}\PY{o}{\PYZca{}}\PY{l+m+mi}{2} \PY{o}{+} \PY{l+m+mf}{100.0} \PY{o}{*} \PY{p}{(}\PY{n}{x}\PY{p}{[}\PY{l+m+mi}{2}\PY{p}{]} \PY{o}{\PYZhy{}} \PY{n}{x}\PY{p}{[}\PY{l+m+mi}{1}\PY{p}{]}\PY{o}{\PYZca{}}\PY{l+m+mi}{2}\PY{p}{)}\PY{o}{\PYZca{}}\PY{l+m+mi}{2}
        \PY{k}{end}
        \PY{n}{res} \PY{o}{=} \PY{n}{bboptimize}\PY{p}{(}\PY{n}{rosenbrock2d}\PY{p}{;} \PY{n}{SearchRange} \PY{o}{=} \PY{p}{(}\PY{o}{\PYZhy{}}\PY{l+m+mf}{5.0}\PY{p}{,} \PY{l+m+mf}{5.0}\PY{p}{)}\PY{p}{,} \PY{n}{NumDimensions} \PY{o}{=} \PY{l+m+mi}{2}\PY{p}{)}
\end{Verbatim}

    \begin{Verbatim}[commandchars=\\\{\}]
Starting optimization with optimizer BlackBoxOptim.DiffEvoOpt\{BlackBoxOptim.FitPopulation\{Float64\},BlackBoxOptim.RadiusLimitedSelector,BlackBoxOptim.AdaptiveDiffEvoRandBin\{3\},BlackBoxOptim.RandomBound\{BlackBoxOptim.RangePerDimSearchSpace\}\}
0.00 secs, 0 evals, 0 steps

Optimization stopped after 10001 steps and 0.021000146865844727 seconds
Termination reason: Max number of steps (10000) reached
Steps per second = 476234.7646370954
Function evals per second = 482282.3413903112
Improvements/step = 0.1993
Total function evaluations = 10128


Best candidate found: [1.0, 1.0]

Fitness: 0.000000000


    \end{Verbatim}

            \begin{Verbatim}[commandchars=\\\{\}]
{\color{outcolor}Out[{\color{outcolor}7}]:} BlackBoxOptim.OptimizationResults("adaptive\_de\_rand\_1\_bin\_radiuslimited", "Max number of steps (10000) reached", 10001, 1.526431919393e9, 0.021000146865844727, BlackBoxOptim.DictChain\{Symbol,Any\}[BlackBoxOptim.DictChain\{Symbol,Any\}[Dict\{Symbol,Any\}(Pair\{Symbol,Any\}(:RngSeed, 1466),Pair\{Symbol,Any\}(:NumDimensions, 2),Pair\{Symbol,Any\}(:SearchRange, (-5.0, 5.0)),Pair\{Symbol,Any\}(:MaxSteps, 10000)),Dict\{Symbol,Any\}()],Dict\{Symbol,Any\}(Pair\{Symbol,Any\}(:FitnessScheme, BlackBoxOptim.ScalarFitnessScheme\{true\}()),Pair\{Symbol,Any\}(:NumDimensions, :NotSpecified),Pair\{Symbol,Any\}(:PopulationSize, 50),Pair\{Symbol,Any\}(:MaxTime, 0.0),Pair\{Symbol,Any\}(:SearchRange, (-1.0, 1.0)),Pair\{Symbol,Any\}(:Method, :adaptive\_de\_rand\_1\_bin\_radiuslimited),Pair\{Symbol,Any\}(:MaxNumStepsWithoutFuncEvals, 100),Pair\{Symbol,Any\}(:RngSeed, 1234),Pair\{Symbol,Any\}(:MaxFuncEvals, 0),Pair\{Symbol,Any\}(:SaveTrace, false)…)], 10128, BlackBoxOptim.ScalarFitnessScheme\{true\}(), BlackBoxOptim.TopListArchiveOutput\{Float64,Array\{Float64,1\}\}(1.755239653260451e-23, [1.0, 1.0]), BlackBoxOptim.PopulationOptimizerOutput\{BlackBoxOptim.FitPopulation\{Float64\}\}(BlackBoxOptim.FitPopulation\{Float64\}([1.0 1.0 … 1.0 1.0; 1.0 1.0 … 1.0 1.0], NaN, [1.61958e-21, 3.34085e-21, 5.68053e-21, 2.31703e-20, 1.09389e-20, 5.71653e-21, 6.68369e-22, 6.89362e-21, 6.52675e-21, 4.057e-21  …  4.30743e-22, 2.6898e-21, 1.26068e-20, 2.27065e-20, 1.24454e-19, 2.36923e-19, 4.26575e-21, 5.41387e-21, 7.61718e-21, 3.85335e-20], 0, BlackBoxOptim.Candidate\{Float64\}[BlackBoxOptim.Candidate\{Float64\}([1.0, 1.0], 24, 1.42957e-21, BlackBoxOptim.AdaptiveDiffEvoRandBin\{3\}(BlackBoxOptim.AdaptiveDiffEvoParameters(BlackBoxOptim.BimodalCauchy(Distributions.Cauchy\{Float64\}(μ=0.65, σ=0.1), Distributions.Cauchy\{Float64\}(μ=1.0, σ=0.1), 0.5, false, true), BlackBoxOptim.BimodalCauchy(Distributions.Cauchy\{Float64\}(μ=0.1, σ=0.1), Distributions.Cauchy\{Float64\}(μ=0.95, σ=0.1), 0.5, false, true), [0.701635, 0.822804, 0.497025, 0.686208, 1.0, 0.588519, 0.922643, 0.497291, 0.494365, 1.0  …  1.0, 0.560735, 0.748291, 0.626623, 0.546719, 1.0, 1.0, 0.950497, 0.911785, 0.80654], [0.73129, 0.0617889, 0.909293, 1.0, 1.0, 0.104108, 1.0, 0.937073, 1.0, 0.980383  …  0.0305664, 0.922521, 0.114515, 0.0876894, 1.0, 0.454371, 0.182781, 0.246594, 1.0, 0.894807])), 0), BlackBoxOptim.Candidate\{Float64\}([1.0, 1.0], 24, 3.43826e-20, BlackBoxOptim.AdaptiveDiffEvoRandBin\{3\}(BlackBoxOptim.AdaptiveDiffEvoParameters(BlackBoxOptim.BimodalCauchy(Distributions.Cauchy\{Float64\}(μ=0.65, σ=0.1), Distributions.Cauchy\{Float64\}(μ=1.0, σ=0.1), 0.5, false, true), BlackBoxOptim.BimodalCauchy(Distributions.Cauchy\{Float64\}(μ=0.1, σ=0.1), Distributions.Cauchy\{Float64\}(μ=0.95, σ=0.1), 0.5, false, true), [0.701635, 0.822804, 0.497025, 0.686208, 1.0, 0.588519, 0.922643, 0.497291, 0.494365, 1.0  …  1.0, 0.560735, 0.748291, 0.626623, 0.546719, 1.0, 1.0, 0.950497, 0.911785, 0.80654], [0.73129, 0.0617889, 0.909293, 1.0, 1.0, 0.104108, 1.0, 0.937073, 1.0, 0.980383  …  0.0305664, 0.922521, 0.114515, 0.0876894, 1.0, 0.454371, 0.182781, 0.246594, 1.0, 0.894807])), 0)])))
\end{Verbatim}
        
    \subsection{JuMP, Convex.jl, NLopt.jl}\label{jump-convex.jl-nlopt.jl}

\href{https://jump.readthedocs.io/en/latest/quickstart.html}{JuMP.jl} is
a large library for all sorts of optimization problems. It has solvers
for linear, quadratic, etc. programming problems. If you're not doing
nonlinear optimization JuMP is a great choice. If you're looking to do
convex programming, Convex.jl is a library with methods specific for
this purpose. If you want to do nonlinear optimization with constraints,
NLopt.jl is a library with a large set of choices. It also has a bunch
of derivative-free local optimization methods. It's only issue is that
its an interface to a C library and can be more difficult to debug than
the native Julia codes, but otherwise it's a great alternative to Optim
and BlackBoxOptim.

    \subsection{Problem 1}\label{problem-1}

Use Optim.jl to optimize
\href{http://al-roomi.org/benchmarks/unconstrained/2-dimensions/58-hosaki-s-function}{Hosaki's
Function}. Use the initial condition \texttt{{[}2.0,2.0{]}}.

\subsection{Problem 2}\label{problem-2}

BlackBoxOptim.jl to find global minima of the
\href{https://arxiv.org/pdf/1308.4008.pdf}{Adjiman Function} with
$-1 < x_1 < 2$ and $-1 < x_2 < 1$.


    % Add a bibliography block to the postdoc
    
    
    
    \end{document}
